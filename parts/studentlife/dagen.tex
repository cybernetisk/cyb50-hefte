\chapter{dagen@ifi - Ifi sin egen bedriftsdag!}

\textit{Red.anm.: I CYB sitt 42-års jubileumshefte skrev Anna Dahl en veldig god tekst om dagens opprinnelse, og vi har valgt å gjengi denne her til glede for nye lesere. Men vi vil også å vite litt mer om dagen@ifi i nyere, noe Karl H. Totland har vært velvillig til å fortelle i kapitlet andre tekst.}

\section{dagen@ifi - litt historie}

\author{Skrevet av Anna Dahl}

Det hele startet faktisk med Verdande (foreningen for kvinnelige Ifi-studenter). I 2003 var foreningsstatusen på Ifi ganske bedrøvelig, med lav aktivitet i de fleste foreningene. Verdande var i en særlig dårlig stilling ettersom alle i styret var ganske enige om at grunnlaget for å ha en egen `jenteforening' på Ifi var et helt annet enn i 1997, da foreningen ble startet. Samtidig var vi klar over at også de andre foreningene slet med liten oppslutning, få (styre)medlemmer og lavt aktivitetsnivå.

22. september 2003 sendte derfor undertegnede `dinosaur', daværende leder i Verdande, en epost til CYB, PING og FUI. Utdrag følger:

\begin{quote}
	Det kan se ut som om viljen til å engasjere seg er synkende blant IFI-studenter. Ut fra det vi har hørt (og ser på de forskjellige websidene), sliter de fleste av oss med rekrutteringen, og det blir holdt atskillig færre arrangementer enn for bare få år siden.
	
	Vi kan sikkert skylde på økt arbeidspress blant studenter, kvalitetsreform og det ene med det andre, men det er nok også på tide å se litt nærmere på hvilket tilbud vi samlet sett tilbyr studentene. Studentforeninger er til for \underline{studentene} -- vårt tilbud skal være med på å gjøre det mer sosialt, bedre, morsommere og lettere å studere ved Ifi. Kan vi egentlig si at vi oppfyller våre egne målsetninger, slik situasjonen er i dag?
	
	Verdande har lenge sunget på siste verset, og vi som sitter i styret i dag, tror ikke at det er <<marked>> for en egen forening for jenter ved Ifi. Derimot tror vi at det burde være fullt mulig å skape et fantastisk miljø for studentengasjement ved Ifi, i samarbeid med dere! Tanker om sammenslåing av studentforeningene på Ifi har såvidt vært luftet tidligere -- nå er det kanskje på tide å snakke skikkelig om det?
\end{quote}

Eposten fikk umiddelbart positiv respons fra leder i CYB, Eirik Munthe, og både FUI og PING sa seg villige til å delta. Etter hvert ble mailen videresendt i alle retninger, og både leder i ProsIT (Knut Johannes Dahle) og Mikro (Omid Mirmotahari) meldte sin interesse.

29. september ble det første «foreningsmøtet» holdt, i styrerommet på Ifi1. Det ble en lang diskusjon - mange var positive til å slå sammen foreningene, men det kom også en del motforestillinger. Situasjonen for de nyere (`nisje'-)foreningene var en litt annen enn for de eldre: de slet av ulike årsaker ikke like mye med rekrutteringen, hadde godt aktivitetsnivå og så dermed ikke det samme behovet. I tillegg kom det opp en del praktiske utfordringer som endringer av vedtekter, valg av navn, diverse generalforsamlinger og nyvalg etc.

Etter hvert ble det klart at full sammenslåing ville koste mer enn det ville smake, men samtidig var det åpenbart at alle ønsket mer samarbeid og samkjøring av aktiviteter og arrangementer. Så dukket idéen opp: hva med et samarbeidsprosjekt i stor skala, der alle foreningene ville få profilert seg overfor studentene og instituttet, og samtidig testet ut hvordan samarbeid kunne fungere? Dette var det stor stemning for - og virket langt enklere å implementere, enn sammenslåing.

En styringskomité bestående av tolv personer ble satt: Dag-Erling Smørgrav, Dagfinn Ilmari Mannsåker, Eirik Munthe, Håvard Moen, Hege L. Pedersen, Kaja Elisabeth Mosserud, Mads Andre Bergdal, Omid Mirmotahari, Per Andreas Norseng, Peter J. Korsmo, Tor Sigurd Mytting og undertegnede. Prosjektet ble først kalt «Den Store Ifi-Dagen (Og Natten) - DSID(ON)», i god informatikerånd.

På det første møtet i styringskomitéen foreslo Hege «Dagen@Ifi» (senere dagen@ifi) som navn på arrangementet, og man fant en dato: torsdag 30. oktober. Det ble planlagt et program med faglige foredrag på dagtid, kombinert med stands i fellesarealene fra 12-16. Foreningene skulle ha felles stand, og alle forskningsgruppene skulle inviteres til å sette opp sine egne. I tillegg ble man enige om å ta kontakt med diverse bedrifter og høre om de kunne være interesserte i å sponse arrangementet, mot å få sette opp egne stands.

Tanken var å holde studentene på Ifi hele dagen, så det var åpenbart at man trengte et trekkplaster som bindeledd mellom aktivitetene på dagtid og kveldstid i tillegg til servering, og mange forslag til gode populærvitenskapelige foredragsholdere kom opp. Eirik Munthe foreslo å hyre inn en stand-up-komiker, og fikk i oppdrag å finne en slik. På kveldstid skulle det så bli fest: DJ/dans og ølservering i foajéen og diverse aktiviteter i auditoriene og grupperommene/korridorene - spill, karaoke, konkurranser og ikke minst den Entrapment-inspirerte laserkorridoren satt opp av Omid Mirmotahari og Mikro.

Sjelden har det vel blitt `blestet' mer aktivt for et arrangement, enn for det første Dagen@Ifi-arrangementet. Siden det var helt nytt, var det høyst usikkert om folk ville synes konseptet var interessant nok til å dukke opp. Helt fra starten av arbeidet i september ble det hengt opp mengder av plakater overalt på Ifi og resten av realfagsbyggene på Blindern, man gikk innom minst én forelesning i hvert eneste fag som hadde forelesninger i perioden september-oktober, mail ble sendt på alle lister, og for første gang ble det gitt tillatelse til å bruke plass på utskriftsforsidene til noe annet enn driftsinformasjon. Instituttet var svært positivt innstilt til arrangementet fra starten av, og hjalp til med både økonomien og blestingen.

Resten er historie, som man sier. Hadde vi hatt noen anelse om hvilken enorm arbeidsmengde prosjektet skulle kreve, hadde vi nok kuttet kraftig ned på ambisjonene (skjønt underestimering er jo gammel tradisjon i IT-relaterte prosjekter). Det var nok ingen i den første dagen@Ifi-komitéen som fikk produsert noe særlig vekttall i oktober det året, og mange endte opp med å døgne på Ifi en god del, men sosialt var det i alle fall! Resultatet gikk imidlertid over all forventning - vi var i utgangspunktet bekymret for om vi ville greie å samle så mye som 100 studenter, men folk strømmet på hele dagen og ble værende på Ifi til langt utpå kvelden. Det førte faktisk til at vi gikk tomme for øl (!) relativt tidlig på kvelden, og det var bare takket være en heroisk innsats fra blant andre Dag-Erling Smørgrav at vi fikk inn nye forsyninger utpå kvelden og dermed unngikk katastrofe.

Det ble ikke noen sammenslåing av andre foreninger enn CYB og Verdande, men til gjengjeld oppnådde man hensikten med prosjektet: det ble vesentlig mer samarbeid og kommunikasjon mellom foreningene etter dette særdeles vellykkede opplegget.

Det at den første Dagen@Ifi fikk så mye oppmerksomhet og ble en slik suksess, gjorde det lett å rekruttere både styremedlemmer og funksjonærer året etter, og siden 2003 har arrangementet blitt større og mer veldrevet for hvert år. Undertegnede `dinosaur' vil gjerne sende en stor takk til alle dere som har ofret tid, krefter, vekttall og studiepoeng for dagen - slitsomt er det, men også helt fantastisk når det går så bra som det gjør!

\section{dagen@ifi i dag}

\author{Skrevet av Karl H. Totland}

I nyere tid har dagen@ifi rettet sitt fokus mot å vokse i tempo med studentmassen ved Ifi. Man har fått flere arrangementer, mer underholdning, flere bedrifter og større fester. I 2015 ble ettermiddagen@ifi arrangert for første gang -- med 4 deltakende bedrifter -- og skulle fungere som en forsmak på dagen@ifi både for deltakere og styret.

I 2016, året etter man arrangerte ettermiddagen første gang, gikk man fra 4 til 10 deltakende bedrifter. Selv om man i 2016 hadde samme antall bedrifter som året før under dagen@ifi, var den voksende interessen for lillebroren et klart tegn på økt interesse blant bedriftene for kompetanse innenfor informatikk.

Under dagen@ifi dette året prøvde man enda et nytt konsept for å sørge for at besøkende forble på bygget til kveldsarrangementet startet, passende navngitt <<mingleområdet>> for at studentene og bedriftene skulle kunne mingle før festen. Konseptet i seg selv hadde god effekt, siden man fikk flyttet folkemengden fra 1. etasje -- hvor man rigget om til bar -- og opp i 3. etasje. Om beliggenheten var en spesielt god idé er annen sak, ettersom det var flere studenter som satt og jobbet med obligatoriske oppgaver mens det pågikk høylytt mingling rett ved siden av.

Dette var også første gangen man gikk bort fra at dagen@ifi måtte holdes siste torsdagen i oktober, og siden har den blitt flyttet nærmere og nærmere semesterstart. Man kan derimot spørre seg om dette egentlig har gagnet studentene i særlig grad, siden det også gir rom for å flytte jobbsøkerfrister tidligere.

I 2017 var antall deltakende bedrifter på ettermiddagen det samme og opplegget ganske likt. Bekk endte i likhet med mange tidligere år opp som hovedsponsor og man bestemte seg for å ta inn enda flere bedrifter i trå med det voksende behovet. Mingleområdet var i år også holdt i 3. etasje, men denne gangen nærmere foreningskontoret med ønske om å holde minglingen litt lenger unna de hardtarbeidende og obligfokuserte studentene.

2018 var første gang foreningen passerte en million i overskudd, noe som kan skyldes at styret bestemte seg for å ta inn 50 bedrifter og fire startups. Det kan hende dette hadde en viss sammenheng med at man dette året hadde en bank som hovedsponsor. Dette overskuddet førte til at flere foreningsaktive nå begynte å snakke om anskaffelsen av en Ifi-hytte i forbindelse med FU sitt nåværende kapital. Under ettermiddagen dette året tok man også inn 13 bedrifter, i tillegg til at hovedsponsor også fikk plass.

Dagen@ifi dette året ble ikke bare holdt før oktober, men i tillegg på en fredag, stikk i strid med tradisjon. I forbindelse med Ifi sine planer om å flytte bibliotekets hovedinngang til glassveggen, ble mingleområdet holdt der denne gangen, men dessverre var ikke dette klart i tide til arrangementet og noe kaos oppsto under både opprigg og nedrigg av barene. Som resultat av at det ble holdt på en fredag var det også merkbart flere bonger i sirkulasjon, med ny rekord satt av hovedsponsor for antall kjøpte.

Av andre relativt nye arrangementer, er det også verdt å nevne Masterkickoff. Dette ble arrangert både for at hovedsponsor skulle få muligheten til å nå flere studenter, samtidig som at nye masterstudenter ved instituttet kunne bli kjent seg i mellom.

Framover ønsker man å gjøre foreningen mer åpen og skape en internkultur som kan løfte arrangementene i større grad. Dagen@ifi står i en særegen stilling med tanke på foreningens natur. Med kun to store arrangementer i året er det ikke like naturlig å skape flyt i arbeidsmengden, og hvert år blir det like vanskelig å fylle et nytt styre. Med det nye vervet sosialansvarlig vil det forhåpentligvis være lettere å skape en større grad av tilhørighet til foreningen og dermed også skape en større grad av kontinuitet.

Det burde også legges mer vekt på å gi tid og rom nok til deltakere og bedrifter, ettersom det å ha mer enn 50 bedrifter i løpet av en dag vil begrense hvor mye oppmerksomhet bedrifter og studenter får fra hverandre. Å spre det utover flere dager -- med litt inspirasjon fra itDAGENE i Trondheim -- kan være en potensiell løsning.