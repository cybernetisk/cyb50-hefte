\chapter*{Om rettskriving}

\setcounter{footnote}{0}

Den oppvakte leser har sikkert fått med seg at vi har varierende former på bestemte ord.

Institutt for informatikk skal forkortes til Ifi, men vi har valgt å beholde formen IFI i navn som IFI-skitur og IFI-POP. Vi har valgt avvikende former fordi navnene de er del av har etablert seg med den formen, og vi følte det ble feil å endre navnet selv om norsk rettskriving tilsier det. 

Cybernetisk Selskab skal i følge norske regler for rettskriving forkortes til Cyb. Men vi har altså valgt å skrive CYB konsekvent fordi det er en innarbeidet form i foreningen.

Vi har prøvd å følge norske tradisjoner for sitattegn ved å bruke tegnene << og >>\footnote{\url{http://www.korrekturavdelingen.no/anforselstegn.htm}}.

Om du er uenige i disse avgjørelsene, så vil vi minne på at når alt kommer til alt, så er det bedre ting i livet å krangle om, og noen sjefsavgjørelser skal man kunne unne seg når man setter sammen en bok.
