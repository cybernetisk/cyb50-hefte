\chapter{Escape}
\label{chap:escape}

\author{Skrevet av Magnus Johansen}

<<Kan du fortelle litt om arbeidet som måtte til for å få Escape til å bli en realitet?>>

Jeg ble stilt dette spørsmålet i anledning jubileum, og svaret kan oppsummeres ganske kort: møter. Massevis av møter og papirarbeid.

I realiteten ble mye av arbeidet rundt selve lokalet som nå er kjent som Escape gjort av studentrepresentanter i samarbeid med de ansatte på Ifi lenge før de aller fleste av oss i det hele tatt visste hva vi ville bli når vi ble store. Mer spesifikt ble avklaringen av lokalet gjort under planleggingen av Ole-Johan Dahls hus, da kjent som Ifi2. Det er nok ganske godt kjent for folk flest at bygget ble planlagt på 90-tallet, og i planleggingsfasen satt representanter for studentmassen, sammen med de ansatte, i et utvalg som la frem ønsket om del av bygget tiltenkt et studentdrevet lokale. Dette lokalet skulle blant annet brukes til pubvirksomhet, noe de andre store fakultetene og instituttene hadde, og som alltid har vært populært i miljøene rundt omkring på Blindern.

I etterkant av dette planleggingsarbeidet gikk Cybernetisk Selskab i dvale, og det `aktive' studentmiljøet på Ifi var ganske lite den gangen. De aller fleste sognet altså til MatNat og holdt til i RF-kjelleren. 

Men i 2007 våknet CYB forsiktig fra dvalen, og i 2009, da spørsmålet kom fra administrasjonen om hvem som skulle ta ansvar for driften av det nye lokalet, meldte CYB sin interesse. Stort sett ble vi ikke utfordret på det, ettersom de fleste var enig i at det ga mening å gi ansvaret til CYB av historiske og sosiale årsaker.

Jeg meldte meg til å ta ansvar for dette, og ble med det konstituert som Kjellermester (den gangen en ambulerende tittel, hverken tittel eller ansvar var avklart eller kjent). I løpet av de neste månedene skjedde det mye, og dette bringer oss tilbake til innledende paragraf: møter. Masse møter og papirarbeid. Kort fortalt ble et kjellerstyre opprettet, og mye av arbeidet innebar informasjonsinnhenting og slektsforskning. Hva gjorde CYB egentlig før dvalen? Hvor mye hadde CYB med Ifi2 å gjøre? Hvilke tradisjoner hadde vi som vi ikke lenger kjente til?

Mange av disse spørsmålene var mer interessante enn det faktiske arbeidet opp mot første åpningsdag. Det viste seg nemlig at å skaffe tillatelse til å drive skjenking i et lokale man ikke eier selv kan være en nokså kronglete og tungvint, spesielt når bygget ikke engang står ferdig. Underveis hadde vi likevel noen morsomme aktiviteter med formål om å lære noe nytt og noe gammelt, og det er tre slike historier jeg tenkte å dele med dere her.

\section{Hvorfor, og hvordan, Escape?}

En av de første utfordringene vi støtte på var navngivning av lokalet. Hva skulle det hete? De andre barene på Blindern hadde navn som satt, og fortsatt sitter, i ryggmargen på folk, og man ønsket den gangen å replikere dette for vårt eget lille sted. <<Ifi-kjeller'n>> fløt ute i eteren allerede, og vi ønsket å ta ballen for å skape snakkis selv før jungeltelegrafen bestemte navnet for oss. Navn som FooBar og Programmerbar (og tusen iterasjoner av <<bar>> med en passende suffix eller prefix i kategorien <<ting som nerder liker og gjør>>) ble allerede så smått kastet rundt, men det var da et raskt voksende studentmiljøet på Ifi, og mange hadde høye tanker om navn.

Samtidig hadde vi jo drevet med slektsforskning om `gamle' CYB, og vi lærte noe ganske artig: dette var ikke første gangen vi som forening hadde drevet bar! I kjelleren på fysikkbygget ble det nemlig drevet en bitteliten bar som den gangen het /local/pub hvor det ble servert flaskeøl fra de stolte lokale masseprodusentene for et bugnende lite miljø av studenter som studerte et fag som er grunnlaget for det som nå kjennes som informatikk. Man oppdaget også at CYB tidligere hadde hatt et verv spesifikt for å drive denne lille barvirksomheten, men der hvor de aller fleste foreningene har en kjellermester som er sjef for de forskjellige pubene, kjørte CYB med tittelen Kjellermogul.

Fristelsen var stor for å ta dette opp på nytt og gjenbruke det som en hyllest til den lange og innflytelsesrike historien til CYB, men var det riktig å navngi et lokale som skulle tilhøre alle uten å snakke med dem først? Vi syntes ikke det, men tittelen Kjellermogul derimot ble nærmest umiddelbart adoptert da jeg syns det var veldig kult, og ingen andre hadde noen innsigelser. Det er tross alt fint å ta vare på historie der hvor det passer seg.

For navnet på puben bestemte vi oss for å ta en runde rundt i studentmiljøet og høre hva folk tenkte, og etter litt samtaler (og flere møter, ettersom studenter er meget glad i møter fordi møter som oftest betyr gratis mat) bestemte vi oss for å avholde en navnekonkurranse. Kort fortalt så ønsket man å inkludere de øvrige foreningene på huset i valget av navn, og dette gjorde vi ved å ta imot nominasjoner fra studentmassene. Fra denne nominasjonslisten stemte CYB internt fram tre toppkandidater som så ble sendt ut til foreningene for avstemning.

Dette førte til mye diskusjon og uenighet, og flere av foreningene hadde motsetninger og ønsket ikke å stemme. En ny konkurranse måtte gjennomføres.

Iterasjon 2 ble noe annerledes: denne gangen fremmet foreningene ett forslag hver, og uken etter stemte alle i fellesskap på den samme listen. Ved en deadlock hadde kjellerstyret i CYB dobbeltstemme.

Til slutt havnet man altså på navnet Escape, et navn som jeg mener å huske Simen Sægrov, da i FUI (men foreslått av kjellerstyret) fant på.

\section{Hvorfor Aass?}

I tiden hvor vi skulle velge en primærleverandør til den nye kjelleren hadde de aller fleste pubene på Blindern Ringnes som leverandør. Mikrobryggeritrenden hadde begynt å få traksjon og folk begynte virkelig å sette pris på mer enn bare `vanlig pils' fra kranene sine.

I kjellerstyret til Cybernetisk Selskab var det flere som hadde utviklet en smak for øl utover det vanlige, og Ifi-studentene var også kjent for å like god øl. RF hadde i all tid hatt en forkjærlighet for Bayer, både som drikkevare og i andre retter; de fleste av oss som har studert på Blindern har nok spist Baffel mer enn én gang.

Med dette i bakhodet gikk vi ut for å se på hva de forskjellige storselskapene hadde å tilby. Ringnes ble raskt satt til siden, hovedsakelig fordi det tidlig ble ytret et ønske om å finne noe som ingen andre hadde. Samtidig viste det seg at flere av de som var aktive i miljøet likte produktene til Aass bedre enn alternativene. Derfra falt valget ganske naturlig, spesielt siden Aass hadde et godt og nært samarbeid med Haandbryggeriet som gjorde at de kunne tilby deres produkter i tillegg til de vanlige tilbudene.

Dog møtte vi litt utfordringer i prosessen. Det er vanlig ved inngåelse av slike avtaler å forhandle priser basert på estimert salgsvolum, men Aass hadde ikke erfaring med å jobbe med studentmiljøer, som har en tendens til å være noe forskjellig fra typiske serveringssteder i Norge. Vi følger rett og slett ikke de etablerte normene som andre steder opererer etter. En studentbar er jo sjeldent åpen på lørdager, som er den typiske norske festdagen. Kombiner dette med en liten gruppe uerfarne studenter som skal til å åpne en pub og forhandlingene startet ikke på sitt beste. Vi ble tilbudt labre priser fra leverandøren, noe som ville gjøre det vanskelig å selge varene til akseptable priser for studenter. Vi hadde ingen interesse av å konkurrere med de andre kjellerne på Blindern, fortjeneste har tross alt aldri vært et punkt å diskutere da alt drives på frivillig basis, men et minimumskrav var å matche prisene hos de andre slik at studentene fritt kunne velge lokaler.

Det var rundt denne tiden hvor vi måtte hente inn bistand fra det øvrige miljøet. Kjellerstyret i CYB hadde lenge fått god hjelp fra venneforeninger med informasjon og søkebistand, spesielt Realistforeningen og Cosa Nostra, men i dette tilfellet måtte vi ha forhandlingshjelp. David Kristensen hadde vært engasjert i CYB litt på sidelinjen, men steppet opp og kom inn i forhandlingsrommet med oss, armert med budsjetter, salgstall og volumer fra RF-kjellerens drift. Dette førte til en god avtale som tilsvarte det de andre foreningene hadde med sine leverandører og vi ble enige om et salgsvolum på 10 000 liter i første omgang. Dette tallet ble sprengt relativt tidlig, og resten er, som de sier, historie.

Som en liten ekstra øltrivia kan jeg fortelle at det ble lagerført én pakke (tolv enheter) med Duvel til enhver tid i i alle fall fire år etter at jeg gikk ut av mine tre år som Kjellermogul, hovedsakelig fordi dette er mitt favorittbrygg.

\section[Signaturdrinker]{Signaturdrinken Lollipop og vår første serveringsrunde}

9. januar 2010 ble Cybernetisk Selskab invitert til å ha sin egen bar/stand på Biørnegildet 2010. Dersom du, kjære leser, aldri har hørt om Biørnegildet så kan jeg fortelle deg at dette er en studentfestival som avholdes hvert tredje år av Realistforeningen for å feire mangfold i vitenskap. Det foregår over én uke, og inneholder for det meste faglig innhold basert på et tema som settes for året, og avsluttes med to eventer som er kjente og kjære for studentmassen: alle pubers fest på fredag og Biørneballet på lørdag.

Dette var første gang i CYBs nyere historie at vi hadde deltatt på Biørnegildet i noen større grad enn enkeltindividers deltagelse, noe som passet meget godt for det nye kjellerstyret i CYB. Dette ble vårt første forsøk på å drive barvirksomhet på egenhånd!

I invitasjonen lød følgende melding fra Elling Hauge-Iversen (arrsjef Biørnegildet 2010): <<Jeg vil gjerne at dere tenker opp et tema for standen deres, slik at jeg kan plassere dere et sted som egner seg best.>>

I forkant av denne invitasjonen hadde vi i kjellerstyret allerede begynt i det små å utforme det første sortimentet som skulle tilbys i Escape ved åpning. I forkant av Biørnegildet hadde vi en siste kveld, kjærlig navngitt <<Miksekveld>>, for å bestemme oss for et lite utvalg av hjemmelagde drinker som vi ønsket å servere som noe unikt. Takket være David Kristensen sin posisjon i Realistforeningens hovedstyre fikk vi lov til å benytte oss av deres lokaler i kjelleren på Vilhelm Bjerkens utenfor normale åpningstider, onsdag uken før Alle pubers fest på Biørnegildet fant sted. I løpet av denne kvelden ble diverse kombinasjoner av alkoholvarer og blandevann testet og smakt på i et rolig, kristelig tempo, ettersom jeg i forkant av dette informerte om følgende:

<<Jeg vil bare si ifra om at dette er i arbeidsformål, så dette er IKKE en subsidiert fyllekveld, dvs. at vi skal SMAKE og prøve, og ikke bli wasted.>>

Dette ble naturligvis overholdt og ingen kan noensinne faktasjekke oss på dette.

Resultatet av denne kvelden ble at to nydelige drinker ble skapt: Lollipop og Peach Potion. Oppskriftene er tatt direkte fra barsjef Henrik Hellerøy under, til glede for nye lesere.

\begin{quote}
	\textbf{Lollipop}
	
	\begin{itemize}
		\item 2/4cl vodka
		\item 6cl grenadine
		\item fyll opp med appelsinjuice
	\end{itemize}
	
	Ha is (knust eller i kuber) i shaker først og shazam.
	
	\textbf{Peach Potion}
	
	\begin{itemize}
		\item 2/4cl koskenkovva peach
		\item Fyll opp med grapesodadings
	\end{itemize}
	
	Ha is (knust eller i kuber) shaker først og shazam.
\end{quote}

På selve dagen til Alle pubers fest stilte kjellerstyret sterkt og med glade funksjonærer, men vi hadde ingen erfaring med faktisk salg og reell drift av en bar med en større mengde kunder på egen hånd. Henrik Hellerøy hadde erfaring fra sin tid med Uglebo og David Kristensen var godt kjent med prosessen fra sin tid i RF, men for det meste var vi alle ganske grønne på det å stå i bar og servere festglade studenter. Til tross for dette holdt vi stand, og leverte en fantastisk god baropplevelse rett ved siden av hovedscenen og dansegulvet i Vilhelm Bjerknes hus, med navnet <<Under Construction>>. 

Temaet var fullstendig mangel på tema og identitet; alle som arbeidet var utsmykket med sikkerhetshjelmer, refleksvester, byggeteip og annet ymse utstyr man forbinder med en byggeplass. Biørnegildets arrangører hadde stilt med tappetårn og miksepult med kum, sitteanordninger og salgsvarer. De bistod også med det tekniske, men vi klarte til og med å skifte våre egne fat. Sidestilt med baren ble det satt opp et tegnebrett med hvite ark hvor besøkende ble invitert til å skrive sitt navneforslag til baren, eller bare en hyggelig hilsen.

<<Under Construction>> var en braksuksess, og en utrolig morsom generalprøve for oss når det kom til å drive skjenking for første gang. De hjemmelagde drinkene våre var også svært populære, og det ryktes at Lollipop vant en pris for beste drink, men etter flere timers søkning i egen e-post kan jeg ikke finne beviser så det får bli med myten.
