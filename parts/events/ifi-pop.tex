\chapter{IFI-POP}

\author{Skrevet av Arne Hassel, godt hjulpet av Thomas Ferris Nicolaisen}

Et viktig arbeid for CYB på 2000-tallet var forskjellige aktiviteter for å bygge opp studentmiljøet. Da man la ned vervet bedriftsansvarlig i 2002 var det som følge av en nedgang i interessen blant bedrifter om å bruke penger på promotering, siden markedet slet etter dot com-boblen. Man hadde klarte å få i gang dagen@ifi i 2003, men dette var bare et arrangement i året, og det var dem som ønsket flere foredrag fra eksterne som ikke nødvendigvis var direkte tilknyttet fag.

Thomas Ferris Nicolaisen var en av dem som følte at det var plass for flere foredrag, og spesielt inspirert var han av JavaBin som hadde holdt noen foredrag i Lille Auditorium på Kristen Nygaards hus. På et av disse sto han opp på slutten foredraget og spurte om det var noen som kunne være interessert i å bidra til flere populærvitenskapelige foredrag på huset. Det var flere som kunne tenke seg dette, og med noen interesserte gjennom studentnettverket hadde han plutselig program for noen foredrag fremover.

17. Januar 2005 var første IFI-POP en realitet, og i løpet av våren fikk man til seks foredrag med representanter fra FreeCol, CoffeeBreaks, Plutolife, Objectnet og Sintef, Dolphin Interconnect Solutions og Bekk consulting. Promotering og penger til mat fikk man hjelp med gjennom instituttet, med gamle Ifi-kjenninger som Omid Mirmotahari og Terje Knudsen som ekstra behjelpelige. Gratis mat har alltid vært et godt trekkplaster for studenter, men promotering var ikke enkelt, og selv om man fikk promo på førstesiden på printersiden, så dukket det som regel opp mellom 30 og 60 studenter.

Ambisjonene var ikke noe mindre de neste semesterne, med åtte foredrag høsten 2005 og ni foredrag våren 2006. Dessverre ser det ikke ut til at man klarte å holde koken, og allerede mot slutten av 2005 anes det at engasjementet var på vei nei. Det hadde blitt laget et eget styreverv for IFI-POP ansvarlig, og selv om den levde videre helt frem til våren 2007 (da som IFI-POP og bedriftsansvarlig) så ble den våren det siste semesteret man hadde egen ansvarlig. (Nok til fordel for Navet, som oppsto på denne tiden.)

Da man hadde fått inn en nytt blod i 2009 prøvde man å gjenopprette IFI-POP, da flere tenkte det kunne være moro med populærvitenskapelige foredrag i tillegg til bedriftspresentasjonene til Navet. Men det viste seg å bli for mye arbeid for et styre som måtte sette inn klutene i arbeidet med den kommende studentkjelleren. (Fotnote: Om ikke undertegnede husker helt feil, fikk man til to foredrag med hhv Dag Langemyhr, som snakket inspirerende om de underliggende teknologiene til fonter, og Roger Antonsen, som snakket om fascinerende mønstre og magiske formler i matematikken.)

