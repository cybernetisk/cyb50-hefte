\chapter[Fordelingsutvalget]{Fordelingsutvalget - studiemiljøet på Ifi sin egen støtteordning}

\author{Skrevet av Arne Hassel og Nikolas Papaioannou}

Som student på Ifi opparbeider man seg en kompetanse som er meget ettertraktet i arbeidslivet. Dette hadde man skjønt mens dagen@ifi utviklet seg til å bli en profesjonell og tradisjonell bedriftsdag, og når man ville starte Navet i 2006 skjønte man også at det ville være fornuftig med en mekanisme som kunne fordele pengene som kom inn på dette arbeidet slik at det kom alle studentene på Ifi til gode. Dette var starten til Fordelingsutvalget.

At studentene på Ifi i 2006 tenkte i disse baner er noe man ikke skal ta for gitt. Det er nok av andre bedriftsdager og bedriftsrepresentanter rundt om i det ganske land som velger å bruke pengene på helt andre måter enn å samle det i en stønadsordning som drives av foreningene tilknyttet undervisningsstedet.

Da ordningen startet var det syv foreninger som satt i utvalget: ProsIT\footnote{Profesjonsstudentenes interesseforening}, FUI, CYB, Navet, Mikro\footnote{Linjeforenigen for robotikk og intelligente systemer}, dagen@ifi og PING\footnote{Program-, Informasjons- og Nettverksteknologisk Gruppe}. Siden da har to foreninger falt fra (ProsIT og PING) og seks nye har kommet til: Defi\footnote{Designforeningen ved IFI}, Digitus\footnote{Linjeforeningen for studieprogrammet Informatikk: Digital Økonomi og Ledelse}, MAPS\footnote{Matematikk, Algoritmer og Programmering for Studenter}, LI:ST\footnote{Linjeforening for Språkteknologi}, ProgSys\footnote{Linjeforening for Programmering og Systemarkitektur (samt Programmering og Nettverk)} og Ifi-Avis\footnote{Foreningen bak magasinet Output}. Disse møtes syv-åtte ganger i året og hvordan pengene skal fordeles generelt og om søknader som kommer inn skal støttes spesifikt.

Opp gjennom tiden har man jobbet for å effektivisere søknadsprosessen, og systemer har kommet på plass for å støtte opp om dette. Dette lever i dag på \url{https://fordelingsutvalget.org} og selve systemet er åpent tilgjengelig på \url{https://github.com/cybernetisk/fordelingsutvalget}, i ekte open-source programmeringsånd.

Man har også fått på plass støtter som er av mer rutinemessig art, som for eksempel at linjeforeningene får 1000 kr som de kan bruke fritt i løpet av semesteret. Dette har gjort at linjeforeningene har mer frihet til å gjøre mindre innkjøp.

At tidene er gode er det ingen tvil om. Ikke sjeldent diskuteres det hvordan pengene kan forvaltes på en god måte, og alt fra fond til hyttekjøp diskuteres. Uansett hvordan man ender opp med å forvalte pengene, så er uansett kjernen av utvalget viktig, nemlig at det skal komme studentene på Ifi til gode. Dette er med på å styrke studiemiljøet og sørge for videre god grobunn for engasjement og gode initiativ.
