\chapter{CYBs fødsel}

\author{Skrevet av Jon E. Dahlen}

\textit{Red.anm.: Denne teksten ble skrevet i anledning CYB sin 25-års feiring. Den gir god innsikt i den første tiden til CYB, og vi håper den gir glede også for nye lesere av denne boken.}

Kilder: Rolf Bjerknes, Ivar Jardar Aasen, Rolf Lind, Alf Hestenes, Nils Christophersen og Cybernetisk Selskabs arkiver.

\section{Bakgrunn}

Våren 1967 tilbød Fysisk Institutt en ny linje, linjen for kybernetikk. Den bestod i 1968 av kursene F51 Informasjonsteori, F52 Signalstatistikk, F53 Reguleringsteknikk og F54 Regnemaskinteknikk. Disse kursene utgjorde tilsammen et tilbud på 1. avdeling, (laveregrad; nå bachelor). Undervisningen i disse fagene var basert på innleide lærerkrefter. Studentene som valgte dette studiet møtte problemer når de ville begynne på hovedfag (nå master). Det var flere grunner til dette. For at et fag eller en linje skal kunne tilby hovedfag må det være minst ett dosentur innen faget, dette var ikke tilfelle for kybernetikk. Det var heller ikke noe undervisningstilbud på hovedfagsnivå i kybernetikk. Matematisk institutt ga på samme tid undervisning i tradisjonelle databehandlingsfag som programmering, undervist av blant annet professor Ole-Johan Dahl. De fleste av studentene ved linjen for kybernetikk fulgte også disse kursene. I løpet av 1968 hadde et tyvetalls studenter tatt fagene som tilhørte linjen for kybernetikk. Man regnet med at det i løpet av 1969 ville være ca. 40 studenter som var klare for hovedfag innen kybernetikk, og disse ville da utgjøre ca. 1/3 av studentene ved Fysisk Institutt. Disse stod så plutselig uten studietilbud når hovedfaget skulle påbegynnes. Det var riktignok et dusin hovedfagsstudenter i 1968, men alle disse hadde eksterne veiledere, hovedsaklig fra Forsvarets forskningsinstitutt (FFI) og Sentralinstitutt for industriell forskning (SI). Det som totalt manglet ved linjen var styring og koordinasjon av hovedoppgavene og faglig kompetanse på studiestedet.

\section{Budsjettforslaget}

Ole-Johan Dahl og Tore Olsen utarbeidet et budsjettforslag for linjen for kybernetikk høsten 1968. Forslaget innebar å opprette en professor II-stilling og et lektorat til kybernetikk. Fysisk institutt søkte også å opprette et dosentur. I sin instilling til budsjett for 1969 foreslo Universitetet å opprette et dosentur til Fysisk institutt, en professor II-stilling og et lektorat tiltenkt linjen for kybernetikk. Stillingene som skulle tildeles ble satt opp i prioritert rekkefølge, med stillingene tiltenkt linjen for kybernetikk på 8. og 9. plass, og det 'ubestemte' dosenturet til Fysisk institutt på plassen foran. Da budsjettbehandlingene begynte i Kirke- og undervisningskomitéen forstod man raskt at Fysisk institutt ikke ville bli tildelt alle de tre stillingene, men regnet dosenturet for 'sikkert'. Tanken var da å overtale Fysisk institutt til å utlyse dosenturet innen kybernetikk.

Da det ble klart at Fysisk institutt ikke ville bli tildelt noen nye stillinger i 1969 begynte frustrasjonen å bre seg blant studentene. Kybernetikkutvalget ble opprettet på initiativ fra studentene den 9. oktober 1968. Utvalgets oppgave var blant annet å vurdere personalsituasjonen ved linjen. Kybernetikkutvalget organiserte møter hvor saken ble diskutert og hva de kunne gjøre for å «vinne tilbake» en av stillingene. Flere alternativer ble diskutert, blant annet om man skulle gå for professor II-stilling eller et lektorat. Valget falt på å prøve å 'vinne tilbake' dosenturet, da dette i utgangspunktet var høyest prioritert. Kybernetikkutvalget innså at det var vanskelig for Fysisk institutt å gjøre noe, men at studentene kunne ha en viss mulighet hvis de tok i bruk lobbyvirksomhet. Oppfatningen var at det ville være lettere å akseptere brudd på tjenestevei fra studentenes side enn fra Fysisk institutt eller fakultetets side.

\section{Lobbyvirksomhet}

Studentene tok da skjeen i egen hånd og utarbeidet et saksdokument med sterke og gode begrunnelser for hvor viktig dosenturet var for forskingsmiljøet og næringslivet i Norge. Dette dokumentet var undertegnet av en rekke aktive studenter, deriblant Rolf Lind, Jørn Archer og Emil Hasle. I dokumentet skriver de blant annet:

\begin{displayquote}
	De eksempler som er nevnt viser at Kybernetikken representerer «know-how» som allerede er nødvendig for en adekvat utnyttelse av våre ressurser, både investeringsmessig og arbeidsmessig.

	Kybernetikkutvalget må derfor gjøre oppmerksom på at andre høyt utviklede industriland arbeider meget intenst innen fagområdet Kybernetikk.

	I Norge er Kybernetikken bare i sin spede begynnelse, men vi kan allerede se at vi også her i landet vil få en rivende utvikling på dette viktige område.
\end{displayquote}

Dette skrivet ble så produsert i like mange eksemplarer som det var medlemmer i Kirke- og undervisningskomitéen. Studentene fant så en politiker fra Høyre som hadde et relativt stort behov for å markere seg. Som ekte lobbyister dro en av dem (Rolf Lind) til Stortinget i forkant av et av komitéens møter for å huke tak i Høyre-politikeren i korridoren. Han ble presentert for saken, og overlevert sakspapirene. Saken ble så lagt fram på møtet, og dosenturet var vunnet tilbake!
Lars Walløe ble ansatt som den første dosent i kybernetikk ved Universitetet i Oslo i 1969.

\section{Samholdet og samarbeidet}

Gjennom kampen for dosenturet i kybernetikk hadde studentene ved linjen skapt sin egen identitet og tilhørighet. De hadde en sak som opptok dem, og som de mente det var verdt å kjempe for. Samholdet og tilhørigheten var noe de kunne bygge videre på, og de bestemte seg for å lage en egen forening for folk med interesse for kybernetikk. Hovedmålene var å informere næringsliv og forskning om kybernetikk, samt å styrke kybernetikken som fag ved Universitetet i Oslo.

Det ble utnevnt et styre av frivillige, engasjerte studenter som tok på seg oppgaven å utforme statutter for foreningen. Det var også en del diskusjon omkring navnet foreningen skulle ta. Man landet tilslutt på Cybernetisk Selskab, som ga inntrykk av å ha en viss tradisjon. Spesielt var B-en i slutten av Selskab viktig, men å skrive kybernetikk med C gir også et visst «gammelt» preg. Styret innkalte så til en konstituerende generalforsamling mandag den 17. februar 1969. Her ble foreningen offisielt stiftet og lovene vedtatt etter en lang diskusjon. Det første styret ble selvfølgelig også valgt. Dette bestod av følgende personer:

\begin{itemize}
\item Ivar Jardar Aasen (leder)
\item Håkon Håkonsen
\item Trond Thue
\item Arne Braathen
\item Eystein Fossum
\item Hans J. Bakke
\end{itemize}

Det viktigste for foreningen i starten var å markedsføre studiet, styrke kybernetikken som fag og å ha et faglig interessant program. Halve styret ble skiftet ut hvert semester (dette gjøres fortsatt) for at folk ikke skulle bli utbrent, men likevel få jobbet fram saker som de brant for.

\section{I ettertid}

Studentene som startet Cybernetisk Selskab var pionerer. De var blant de første i Norge som studerte kybernetikk. De kjempet med pionerånd for faget sitt og for muligheten til å ta hovedfag i kybernetikk. De var uten tvil farget av tiden de levde i, selv om Paris var mye lenger unna for realistene enn for samfunnsviterne, og selv om de kanskje ikke oppdaget hvilken tid de hadde levd i før i ettertid. «Avstanden» til Paris skyldes muligens det faktum at matematisk naturvitenskapelige fag ikke er gjenstand for like følelsesladde og dyptgripende diskusjoner som samfunns- og filosofifag. Kanskje er dosenturet de 'vant' også et resultat av tiden, kanskje var det lettere å bli hørt som student etter opptøyene i Paris?
