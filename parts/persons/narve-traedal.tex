\chapter{Narve Trædal -- en engasjert Ifi-entusiast}

\author{Skrevet av Dag Langmyhr og Arne Maus}

\begin{wrapfigure}{r}{5cm}
	\centering
	\includegraphics[width=0.3\textwidth]{images/narve-traedal/profil.png}
	\label{fig:narve-traedal}
	\caption{Illustrasjonsbilde av Narve Trædal.}
\end{wrapfigure}

Narve Trædal har som kontorsjef bidratt til å sette sitt preg på Ifi, en arv han har holdt i hevd fra hans forgjenger Beth Hurlen.

Da Narve kom til Ifi, var det en spennende administrator som kom. Han hadde vært medlem av SUF(m-l) og leder av Studentersamfunnet, men mest kjent var nok <<Trædal-saken>>. Da Narve avtjente militærtjeneste som sambandsmann i Bodø, kom han over flere hemmelige Nato-telegram hvor det framgikk at militæret øvde mot venstreorienterte og fagforeninger. Disse sendte han til Klassekampen som offentliggjorde dem på forsida. Militæret etterforsket dette, og Narve tilstod så etter flere avhør. Han ble fengslet i tre uker med brev-og-besøksforbud. Saken fikk stor politisk oppmerksomhet, støttekomiteer ble opprettet, bok ble skrevet og saken ble avsluttet med at Narve ble dømt til 60 dagers betinget fengsel – en dom som siden ble opphevet av Høyesterett. Det skal bemerkes at ingen på Ifi synes at det han gjorde, var galt.

Etter dette var Narve student på Blindern på HF og SV. Etter det jobbet han som studieveileder på HF og SV, var leder i NTL på UiO, lønningssjef i personalavdelingen mm. og ansattrepresentant i Det akademiske kollegium. Så kom han til Ifi, først som nestleder i administrasjonen i 1992, og så som kontorsjef fra 1998. Riktignok trodde han at han skulle roe seg ned i 2012 da han fikk jobb som spesialrådgiver i fakultetsadministrasjonen; men det var såpass kjedelig at han kom tilbake til oss i 2013 og forble her i sin gamle jobb som kontorsjef til han ble pensjonist i 2016.

Narves innsats på Ifi har vært vesentlig. Først og fremst var han en hyggelig og ryddig person som fikk ordnet mye. Viktig var også at Narve hadde et godt humør. Hans gjennomgang på julebordene om <<rikets tilstand>> var stor humor og julebordets høydepunkt. Han var også en til tider kontroversiell person med synspunkter som ikke ble delt av alle. Narve syntes og synes at Ifi burde bli et eget fakultet. Det var ikke bare en mening, men han jobbet også for dette, noe som ikke var populært hos andre institutter på MatNat med færre studenter enn Ifi. Han synes også at vi kan greie oss med to nivåer administrasjon på UiO – gjerne uten fakultetene.

For studentene sto han på slik at Ifi fikk en studentkjeller med skjenkerett i det nye Ole-Johan Dahls hus. Dette er et arbeid som Cybernetisk Selskab har vist å sette pris på ved å tildele han CYBs æresmedlemskap våren 2014.

Han har også prøvd å få vitenskapelige ansatte til å møtes regelmessig i en Faculty Club, men lykkes bare med det like før jul når han kan lokke med julekake med kviteseidsmør og brunost.

I dag er Narve en grei kontorsjef emeritus som sitter og skriver Ifis historie hvor han blant annet vil se på om visse miljøer som Buerommet i Ifi-I var mektigere enn Ifis styre. Vi som satt på Buerommet, kan bare frykte Narves snarlige dom.
