\chapter[Terje Knudsen]{Terje Knudsen – grunnbjelken i Ifis IT-drift og byggeprosesser}

\author{Skrevet av Narve Trædal}

\begin{wrapfigure}{l}{7cm}
	\centering
	\includegraphics[width=0.45\textwidth]{images/terje-knudsen/profil.png}
	\label{fig:terje-knudsen}
	\caption{Illustrasjonsbilde av Terje Knudsen.}
\end{wrapfigure}

Terje Knudsen startet som IT-ingeniør ved Ifi i 1991, samme år som han tok sin hovedfagseksamen i digitalteknikk ved Ifi. I 1998 ble han leder for Driftsseksjonen ved Ifi, en stilling han hadde fram til 2015. Da gikk han over i 80\% stilling som ansvarlig for interiørprosjektet i planleggingen av Life
Science-prosjektet. Han har vært en viktig person i planprosessene fram til første byggebevilgning ble gitt i 2018, og vil trolig fortsatt være engasjert gjennom hele byggeprosessen, som er antatt avsluttet i 2024. I tillegg vil han sannsynligvis også bli brukt av fakultetet i planleggingen av hvordan
de resterende bygningene på Nedre Blindern skal brukes etter at aktiviteten der er flyttet inn i Livsvitenskapsbygget.

Hans generelle interesse for arkitektur og kunst er en god ballast å ha med seg i disse prosjekteringsprosessene. Hans tjueårige virksomhet som sakkyndig dommer i tingrett og lagmannsrett har også gitt ham verdifull juridisk prosesskompetanse.

Det som først og fremst preger Terje, i tillegg til hans faglige allsidighet, er hans grundighet og hans stå-på-vilje. Det gjelder i alle sammenhenger, fra relativt enkle flytte- og installasjonsprosesser til krevende laboratorieetableringer. Innflyttingen i Ole-Johan Dahls hus medførte en helt ny dimensjon i satsing på AV-utstyr, men også med en del krevende driftsproblemer. Terje har stått på i et utall timer for å sørge for at studentene skulle bli minst mulig skadelidende. Han ledet en driftsstab på over ti fast ansatte ingeniører, samt en stor stab med terminalvakter, som etter hvert
har gått over til å bli erstattet av resepsjonstjenester og laptop-hjelp, samt at USIT har overtatt en del av det grunnleggende IT-brukeransvaret.

Helt fra starten i 1991 var han sentral i de mange ombyggings- og flytteprosessene ved Ifi\footnote{Se kapitlene \ref{chap:ifi1} og \ref{chap:ifi2}}, men også i flere andre IT-etableringer ved fakultetet. Men det var særlig i perioden fra 2005 og framover at hans innsats var helt uvurderlig for Ifi. Professor Tor Sverre Lande hadde vært Ifis hovedansvarlige i innledningen av planprosessen for Ifi II, men da han dro på ett års forskningsopphold, var det naturlig at Terje Knudsen tok over hans plass som Ifis sentrale representant. Han ble således den ansatte ved Ifi som i størst grad kom til å prege både planleggings- og byggeprosessen Da budsjettrammene for byggeprosjektet ble redusert, var det han som sikret at Ifis interesser ikke ble skadelidende. I selve bygget satte han også varige spor etter seg. Selvsagt i etableringen av IT-utstyret, men også i rom-programmet. Mest synlig er kanskje det han gjorde på navne-fronten, da han fikk oppkalt alle undervisningsrom etter eldre programmeringsspråk. Det kan virke fremmed for faglig ukjente, men framhever informatikkfagets historiske sjel. Den kompetansen han erhvervet seg i denne tiden, og de evner han viste seg å ha for planlegging\slash prosjektering og byggeprosessoppfølging, har som nevnt også etterpå blitt høyt verdsatt av ledelsen i UiOs Eiendomsavdeling.

Det engasjementet og arbeidet han har gjort for Ole-Johan Dahls bygg er også grunnen til at CYB våren 2014 valgte å gi han CYBs æresmedlemskap.

I den stillingsbrøken, 20\%, han har igjen ved Ifi, er arbeidet hans først og fremst til utvikling og drift av store installasjoner av audiovisuelt utstyr, ved hele fakultetet, men også generell systemprogrammering, utvikling og drift av instituttets øvrige utstyr.
