\documentclass[../main.tex]{subfiles}

\begin{document}
av John Bothner

\subsection{De første årene}
CYB i dag er på mange måter svært lik CYB slik det utviklet seg etter ca. 2-3 år. Helt fra starten var CYBs hovedgeskjeft foredrags-virksomhet og organisering av ekskursjoner til bedrifter eller insti-tusjoner.
I tillegg hadde CYB da enkelte funksjoner som i dag ivaretas av Fagutvalget ved Ifi. Som vi var inne på i forrige artikkel, så ble jo CYB opprettet med formålet å ivareta kybernetikkstudentenes interesser. For eksempel ble det våren 1971 arrangert et fellesmøte med fysisk fagutvalg om det å ha ekstern hovedoppgave. Et annet eksempel på en "fagutvalg-rolle": våren 1972 holdt CYB en orientering om pensum i de forskjellige kybernetikk-kursene.

Innledningsvis kan det være morsomt å presentere ordet kybernetikk, slik de første kybernetikere i CYB brukte det. Utdrag fra en pre-sentasjon av CYB høsten 1969:
"For dem som ikke er kjent med begrepet kybernetikk, nevner vi stikkord som: databehandling, automatisk styring, og kontroll av ulike prosesser, simulering (av prosesser). Kybernetikken kan således sies å berøre alle fagområder."

Til å begynne med ble møtene stort sett holdt på onsdager, men fra 73-74 gikk man over til å holde dem på torsdager.

\subsection{Kan datamaskinen erstatte politikerene?}
CYBs kanskje største mediesuksess gjennom tidene er debatten "Kan data-maskinen erstatte politikeren?" 5.10.69 ble det organisert en panel-debatt hvor spørsmålet var i hvilken grad datamaskiner kunne forutsi de samfunns-messige utslag av politiske avgjørelser. Ville datamaskinen kunne overta rollen som politisk rådgiver? CYB hadde i anledningen trykket opp 3000 løpesedler. 250 tilhørere møtte opp i Fredrikke, hvor møtet fant sted. Panelet bestod av 2 politikere, 2 samfunnsvitere, og 2 kybernetikere/databehandlere:

\begin{itemize}
\item Statsråd Helge Seip
\item Stortingsmann Toralf Westermoen
\item Forskningsleder Finn Solie
\item Dr. philos Jens A. R. Christophersen
\item Amanuensis Lars Walløe
\item Professor Ole Johan Dahl 
\end{itemize}

Møteleder var Per Øyvind Heradstveit, programsekretær i NRK. Innledningen ble holdt av professor Jens Balchen, NTH, via telefon, fra Trondheim. Han kunne ikke ta flyet fra Værnes, pga. tåke. Det var for øvrig han som ga CYB ideen om en slik debatt. Under debatten fremhevet statsråd Helge Seip datamaskinens mulighet til å eliminere demagogi, siden den er egnet til å konfrontere oss med fakta og ut i fra gitte forutsetninger fastslå virkninger av tiltak vi prøver å regulere samfunnsprosesser med. Eksempelvis kunne man da unngå krangel om virkningene av 1/2\% renteforandring, eller endring av moms-nivået. Professor Dahl mente at data-maskinens viktigste oppgave i politikken var å gjøre informasjon mer tilgjengelig.

Resultatet av debatten ble fyldige reportasjer i Dagbladet (med bilde fra møtet) og i Aftenposten. Sistnevnte tok til og med emnet opp i lederen under tittelen: "Modern Times". Aftenposten konkluderte i sin reportasje med at ingen ville erstatte politikerne med maskiner, men de fleste mente datamaskinen kunne være til stor hjelp for politikerne.

Man var tydeligvis flink til å mønstre medienes oppmerksomhet de første årene. Også 13.10.71 opplevde man bra dekning av Aftenposten etter møtet med temaet "medisinsk databehandling".
Av spesiell interesse merket forfatteren seg styrets beslutning om å utlyse en konkuranse for en kybernetikksang (23.2.72). På styremøtet 26.4.72 ble vinneren kåret: Osmund Fiskaa! Resultatet kan beskues i kapittelet "Sanger vi gjerne synger".

Allerede tidlig i CYBs virke var tilknytning til næringslivet blitt viktig. Man arbeidet med å få bedriftsmedlemmer (les: sponsorer). Vedtekter for bedriftsmedlemskap ble vedtatt på generalforsamlingen 5.5.71. På samme generalforsamlingen kunne man glede seg over støtten fra "Oslo-bryggeriene": 4 kasser øl!

Tradisjonen med øl og spekemat til generalforsamlingen startet tidlig, første gang nevnt i referatene for 26.10.71 Antagelig er dette en tradisjon man har arvet fra Fysikkforeningen.
CYB var tidlig ute med kvinnefrigjøring, høsten 1971 kunne CYB skilte med Kaja Huster, sin første kvinnelige leder. Hun ble ikke den siste.

9.4.73 var temaet "Kybernetisk Krigføring". Kveldens foredragsholdere var professor Johan Galtung og forskningssjef Erik Klippenberg fra FFI. Etter referatene å dømme en bra "innledning" til USAs "high-tech" krig mot Saddam Husseins Irak nesten 20 år senere.

\subsection{Et eget institutt}
Tanken om et nytt institutt hadde vært fremme i flere år. I 1974 var situasjonen den at kybernetikk hadde 1/3 av fysikkstudentene, men bare 1/20 - 1/30 av lærerne. En noe lignende situasjon ble opplevd på databehandling på matematisk institutt. Dette ble drøftet på møte 7.3.74. På spørsmål fra lærerne kom det frem at kybernetikkstudentene ikke følte seg spesielt knyttet til den øvrige fysikkaktiviteten, eller til Fysisk institutt. Man så på fysikk bare som en av mange anvendelsesområder for kybernetikk. Videre ble likhetene og forskjellene i aktivitetene kybernetikk- og databehandlingsgruppene drev med drøftet. Cyb ble oppfordret til å få i gang en gruppe som på uformelt grunnlag skulle orientere seg om situasjonen med hensyn til sammenslåing av kybernetikk og databehandling til et evt. nytt institutt. Senere referater bekrefter at kybernetikkstudentene ikke hadde noen særlig "fysikkidentitet": Det ble bemerket at det var uheldig at fysisk fagutvalg delte ut lesesalsplasser til kybernetikere, da "kontakten mellom disse to grupper ikke er altfor god".

På ekstraordinær generalforsamling 11.9.75 orienteres det om innstillingen av 12.6.75 om den eventuelle sammenslåingen av kybernetikk og databehandling til et institutt. Jonassen, som var den store drivkraften, uttrykte sin store forbauselse over at komiteen hadde kommet frem til en enstemmig innstilling: den anbefalte en geografisk og faglig samling av databehandling og kybernetikk, til et institutt, informatikk. Det ble på generalforsamlingen spesielt spurt om hva som skulle skje med digitalteknikk som ville ligge i grenselandet mellom informatikk og fysikk. Man mente dette problemet ville bli løst ved et godt samarbeid mellom de to instituttene.

25.10.76 holdt for første gang CYB og Fagkritisk Gruppe ved databehandling ved matematisk institutt, felles styremøte. Møtet er betegnet som historisk i referatene. På dette møtet ble det bestemt at CYB skulle skaffe 2, og FKG 3 studentrepresentanter til det kommende instituttrådet. Samtidig ble det bestemt at man skulle opprette et fagutvalg på informatikk. Fagutvalgets rolle skulle være: opprette kontaktpersonordning, se på undervisningssituasjonen og ressursbehov og administrere lesesalsplasser på informatikk Fag-utvalget skulle bestå av 5 representanter. Det ble vedtatt at CYB skulle skaffe 2, og FKG 3 representanter.

Kollegiet ved Universitetet i Oslo vedtok 1.10.76 opprettelsen av Institutt for informatikk, med virkning fra 1.1.77. For CYBs vedkommende skjedde dette formelt under generalforsamlingen 11.11.76. Dette skjedde stort sett ved at "Fysisk institutt" ble endret til "Institutt for Informatikk" i de aktuelle avsnittene i CYBs lover (med 18 mot 7 stemmer).

\subsection{Ekskursjoner}
Servomøtet/uka i Trondheim ble besøkt av CYB første gang høsten 1969. Det ble fort en fast tradisjon for CYB. Før et år var omme var ekskursjoner til NTH, til Bergen (Christian Michelsens Institutt) og til Kongsberg Våpenfabrikk (nå Norsk Forsvarsteknologi) blitt gjennomført. Ambisjonsnivået var også på topp: Allerede første året drøftet man muligheten for en tur til Sovjet sammen med Fysikk-foreningen. Det ble det dessverre ikke noe av.

Sommeren 1984 iverksatte CYB kanskje sitt hittil største løft. Da dro 9 cybbere på USA-ekskursjon. Allerede tidlig høsten før hadde man begynt arbeidet med forberedelsene. Ved en del arbeid klarte man å få finansiert turen med støtte fra diverse sponsorer/instanser. Turen var rettet mot bedrifter som drev med "hardware"-utvikling, så det var mest instituttets digitalteknikere som ble med. Turen gikk til diverse bedrifter på den amerikanske øst- og vestkysten (Silicon Valley), men man fikk også med seg MIT og Stanford University.

Også høsten 1988 ble det organisert en USA-tur i regi av CYB. Det var da ca. 17 stykker som dro over dammen, med støtte fra universitetet, instituttet og næringslivet. Det var stort sett hovedfags-studenter som ble med. Denne turen var mindre "hardware"-vinklet. Bl.a. besøkte man AI-labben og media-labben på MIT, og Thinking Machines. Og så dro man over til vestkysten hvor man bl.a. besøkte Apollo, Amdahl, og Sun.

Senere utenlandsturer har for CYBs vedkommende begrenset seg til et par turer til Ålborg, ingen har tatt opp hansken etter "USA-farerne" i 1984/88. Muligens skyldes det et hardere økonomisk klima for næringslivet?

14.1.87 feiret man instituttets 10-års jubileum med diverse foredrag, fest og minirevy på terminalstua i fysikkbygget!

Sommeren 1989 var informatikkbygget ferdig. Og så fin som vi syntes den var, vår egen "datadal" i Gaustadbekkdalen! Men plassproblemene meldte tidlig. Mye vil ha mer! Og så måtte vi jo "aksjonere" litt, til fortvilelse for sporveiene, før myndighetene endelig bygget gangbroen "vår". Det ble en "folkesport" i å trosse sporveiens hindringer. Og CYB flyttet med på lasset til nybygget, først var CYBs kontor i 2.etage (ved luka), før dagens mer permanente løsning i "nordfløyen" i 1. etasje.

Så oppdager forfatteren et brev i arkivet fra 1990 undertegnet av han selv (til driftsavdelingen på universitetet)... og minner fra generalforsamlingen våren 1990 kommer tilbake... Det var ikke en kjedelig kveld, nei... Først havnet en serviett oppå et telys, da gikk "merkelig nok" brannalarmen. Da kom selvsagt brannvesenet, selv om vi ringte og forklarte at det var falsk alarm. Når brannvesenet hadde dratt fortsatte festen ved godt mot. Stemningen var fortsatt høy, ja så høy at 4 av gutta (undertegnede ikke inkludert!) iverksatte litt bading "i nettoen" i fontenen utenfor kantina (det er jo tross alt en fin sklie...).

CYB fikk senere en regning på 3000 kroner for falsk brannalarm. Vi skrev et pent brev til driftsavdeling hvor vi bedyret vår uskyld, og la spesiell vekt på at vi var en fattig forening med lite penger å hoste opp. Det siste argumentet var muligens utslagsgivende, vi hørte i alle fall aldri noe mer om den saken.
\end{document}