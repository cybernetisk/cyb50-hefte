\chapter*{Forord}

\begin{wrapfigure}{r}{3cm}
\vspace{-180pt}
\centering
\includegraphics[width=0.2\textwidth]{images/preface-introduksjon/arne-hassel.png}
\captionsetup{labelformat=empty}
\caption{Skrevet av\\Arne Hassel}
\end{wrapfigure}

Cybernetisk Selskab (CYB) er en formidabel forening på Institutt for informatikk (Ifi). I femti år har foreningen jobbet for å bedre forholdene for studenter, være det for dem som gikk linjen Kybernetikk eller for alle studenter på instituttet. Hva dette har betydd i praksis har endret seg gjennom tidene, fra å være rettet mot mer faglig og studentpolitisk innhold, til mer sosialt og kulturbyggende.

Denne boken er et forsøk på å beskrive historien til CYB, Ifi og studentmiljøet på instituttet. Hovedvekten ligger på førstnevnte: dette er tross alt CYB sin jubileumsbok, men det blir en mangelfull helhet av CYB om man ikke også inkluderer det miljøet som foreningen har vært del av.

Boken er delt i fire deler: Første delen tar for seg den overordnede historien til CYB og Ifi, og prøver å beskrive de store endringene som har skjedd, med artige anekdoter her og der. Den andre delen fokuserer på turer og arrangementer som har vært -- eller har blitt -- tradisjoner i CYB. I tredje del har vi tatt for oss foreninger og deler av studentmiljøet på Ifi som CYB gjerne har hatt en fot innenfor. Og i siste del har vi skrevet om personer som har hatt en ekstra betydning for CYB opp gjennom tidene, da med fokus på æresmedlemmer og sentrale skikkelser på Ifi ellers.

Ingen av disse delene er uttømmende, og det er selvfølgelig mer som har skjedd i CYB og på Ifi enn det som er beskrevet i denne boken. Noe av begrensningen ligger i hva den kollektive hukommelsen husker, mens annet ligger i hva man velger å huske. Håpet er selvfølgelig at boken skal være mest mulig etterrettelig, og i det øyemed er det også en nettside tilgjengelig, som ligger på \url{50.cyb.no}. Vi inviterer alle til å besøke den for å gi oss beskjed om eventuelle forbedringer som kan foretas.

CYB har vært mye gjennom tidene, men uansett hva det har vært har foreningen alltid vært et speilbilde av de aktive som har drevet den, og i kjernen av alle de historiske speilene er det en ildsjel som har brent varmt og godt. Ja, det har vært mørke tider iblant, og det vil det også være fra tid til annen fremover. Men om vi aldri glemmer den varmen som er engasjementet i CYB og på Ifi, så lever jeg i beste tro om at studentmiljøet på instituttet vil fortsette å være blant de aller beste å engasjere seg i.

\begin{description}
	\item[Redaktør] Arne Hassel
	\item[Sjefsdesigner og \LaTeX{}] Veronika Heimsbakk
	\item[Designere] Silje Merete Dahl, Katrine Gunnulfsen
	\item[Korrekturlesere] Lee Kvåle, Trine Høyås, Eivind Hauger, Karl Hole Totland
\end{description}

\section*{Takk til}

Denne boka hadde aldri blitt til uten hjelp fra samtlige. Takk til Fordelingsutvalget som valgte å støtte publikasjonen av boken\footnote{Strengt talt, forhåpentligvis velger å støtte boken. Innen den er i trykk har ikke FU formelt bestemt å gi støtten enda.}, takk til dem som har skrevet tekster til boka, og takk til dem som har vært med å utforme og korrekturlese boken slik at den blir så god som den kan bli.

En spesiell takk til alle forfatterne våre som har kommet med bidrag. Narve Trædal har vært et arbeidsjern uten like og vi setter utrolig stor pris på hans bidrag til boken. Ellers har følgende personer bidratt med tekster (i alfabetisk rekkefølge): Andreas Nyborg Hansen, Anna Dahl, Arne Maus, Egil Øvrelid, Geir Arild Byberg, Karl Hole Totland, Magnus Johansen, Morten Moen, Ole Christian Lingjærde, Nikolas Papaioannou, Suhas Govind Joshi, Thao Tran, Thomas Ferris Nicolaisen og Torgeir Lebesbye. Takk også til dem hvis tekster vi gjenbrukte fra CYB25 jubileumshefte: Jon Erlend Dahlen og Rolf Bjerknes (og takk til hans familie som lot oss gjenbruke teksten).

I forbindelse med å samle inn tekster har vi prøvd å ta kontakt med 315 av 384 alumni. Av disse har 119 svart på et eller annet vis, og av disse igjen har 33 gitt bidrag av varierende lengde. Vi har også fått hjelp fra andre tidligere studenter på Ifi ifb.~innsanking av informasjon om andre foreninger. Utover forfatterne er disse (i alfabetisk rekkefølge): Anahita Panjwani, Astrid Elisabeth Jenssen, Birgitte Kvarme, Bodil Bye Larsen, Carl-Magnus Lein, Dag Tungvåg, Daniel Chaibi, Dennis Norheim, Eli Berge, Hans Christian Palm, Hege Kolbjørnsen, Henrik Lilleengen, Håkon Kløvstad Olafsen, Ingrid Aarnes, Jan Ingvoldstad, Jon Ølnes, Knut Erik Borgen, Kristin Skar, Kurt Nilssen, Mats Astrup Schjølberg, May-Lis Farnes, Nils Petter Sundby, Odd Harry Ophaug, Pål Taraldsen, Silje Merethe Dahl, Siri A.M. Jensen, Siw E. Møller-Pettersen, Steinar Kjærnsrød, Steinar Meen, Svein Bøe, Terje Dahl, Thomas Ferris Nicolaisen, Tor Dokken, Tor Ivar Johansen og Vibeke Stoltenberg.

\addcontentsline{toc}{chapter}{Forord}
