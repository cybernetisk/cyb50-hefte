\chapter[Elisabeth Hurlen]{Elisabeth Hurlen – Ifis <<mor>> i krevende tider}

\author{Skrevet av Narve Trædal}

\begin{wrapfigure}{l}{7cm}
	\centering
	\includegraphics[width=0.45\textwidth]{images/elisabeth-hurlen/profil.png}
	\label{fig:elisabeth-hurlen}
	\caption{Illustrasjonsbilde av Elisabeth Hurlen.}
\end{wrapfigure}

Elisabeth Hurlen, eller Beth, som alle kalte henne, var ansatt som leder av administrasjonen ved Ifi fra før instituttet ble opprettet i 1977, og fram til hun fylte 70 år, i 1998.

Hun hadde hovedfag i kjemi fra 1955, og var gift med professor Tor Hurlen ved Kjemisk institutt. Begge døtrene deres tok også etterhvert hovedfag ved Ifi. Sammen med en en eldre kontorsekretær utgjorde hun duoen som var de enseste administrasjonsansatte ved et fag med eksplosiv studentvekst, hvor det alltid var behov for flere undervisningsrom, undervisere og veiledere. Det ble også etablert en stor stab av timebetalte studenter i stillinger som gruppelærere og terminalvakter. I tillegg provisoriske tilleggsarealer, som Brakke I, der hvor Helga Engs hus nå står, da instituttet holdt til i Fysikkbygningen. Senere kom også <<Brakka>>, nord for Informatikkbygninen, der MiNaLaben nå ligger. Før obliger skulle leveres kunne det være lange køer utenfor disse lokalene. Og det var alltid mange studenter der, natt og dag, og også i alle helger, såvel som påske som til jul.

Hennes første ti år var således preget av nærmest konstante flytteprosesser; først fra Matematikkbygningen til Fysikkbygningen, og deretter til Informatikkbygningen i Gaustadbekkdalen. Og hvert sted var det bytte av undervisningrom, kontorer, og ikke minst, etter flyttingen til Informatikkbygningen - ombygginger. Hovedfagslesesaler, som det i utgangpunktet var en del av, måtte etter hvert tas i bruk til den voksende staben. Det kunne ha utløst konflikter, men Beths utpregede samarbeidsevner hindret eventuelle tilløp til slikt. Studentene forsto også at det ikke var noen vei utenom, dersom de skulle få ansatt de lærerne og det it-driftspersonale de trengte . Instituttet fikk etter hvert mer saksbehandlerhjelp, i form av forværelsestjeneste, studentadministrasjon og økonomi- og peronsalfunksjon. For Beth stilte det større krav om mer lederansvar, med rettledning og opplæring. Det taklet hun godt. Nøyaktighet og humør var stikkord. Hun samarbeidet også hele tiden vennlig og konfliktfritt med administrasjonen på fakultetsnivået.

I de første årene etter hennes ansettelse, var det henne studentene kom til med det de måtte ha av spørsmål knyttet til studiene og studiehverdagen sin. Etter at administrasjonen vokste, og i tråd med det skjedde en foryngelses- og kompetanseheving, hadde hun mindre direkte kontakt med studentene. I de siste 15 årene ble hun nok oppfattet mer som <<mor>> av de ansatte enn av studentene. De forholdt seg mer til <<sine>> saksbehandlere, og til vitenskapelig ansatte i ulike verv; særlig leder og medlemmer av Undervisningsutvalget.
