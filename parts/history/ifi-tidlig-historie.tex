\chapter{Ifis tidlige historie}

\author{Skrevet av Narve Trædal}

\section{Røttene}

Datafaget hadde sin spede begynnelse på første halvdel av 50-tallet. Det var lenge en  aktivitet for spesielt interesserte, men blant disse spesielt interesserte fantes Ole-Johan Dahl og Kristen Nygaard som fikk kjennskap til de nye datamaskinene i Norge som ansatte ved Forsvarets forskningsinstitutt (FFI) på Kjeller. 

Ved Universitetet i Oslo startet faget ved Fysisk institutt, hvor den 'hjemmelagde' datamaskinen 'Nusse' ble tatt i bruk i 1953. Maskinen tilhørte egentlig Sentralinstitutt for industriell forskning (SI), som leide rom i kjelleren i Fysikkbygget, men den ble også brukt av universitetet. 

Som undervisningsfag vil mange betrakte professor Selmers seminar ved Matematisk institutt på midten av 50-tallet som et startpunkt, og Harald Keilhau holdt kurs i programmering ved FFI i 1958. Røttene til faget lå innen disiplinene matematikk og fysikk/ingeniørfag, men det var en tredje komponent som dessverre var dårlig representert ved UiO: organisasjons- og administrasjonskunnskap, senere bedre kjent som systemarbeid. Denne ble senere konstituerende for fagtilbudet ved instituttet. 

Opptakten til det som i senere tid ble Universitetets senter for informasjonsteknologi (USIT), det som da var EDB-senteret, kom noen år senere, ved at universitetet i 1960 kjøpte inn en Wegematic 1000.

\section{Redskapsfag eller universitetsdisiplin}

På slutten av 60-tallet og utover i 70-årene var det en nærmest eksplosiv økning i interessen for EDB-utdanning blant studentene, et tydelig uttrykk for at samfunnets interesse for feltet økte sterkt. I tillegg er datafeltet sentralt i alle offentlige utredninger om teknologisk satsing fra 1965 og utover. Når det gjaldt akademisk, forskningsbasert utdanning, var imidlertid interessen noe mindre i politiske kretser. Her ble det hovedsakelig fokusert på kortvarig redskapspreget utdanning, som førte til at universitetetene (og NTH) lenge sto alene om å se nødvendigheten av å åpne for eksperimentell naturvitenskapelig forskning på faget.

Dette var for så vidt naturlig; den utdanningspolitiske dagsorden i slutten av 60-årene og framover var sterkt preget av utredninger for å etablere et nytt system med distriktshøgskoler (DH). Ottosen-komiteen la opp til at den framtidige satsingen på postgymnasial utdanning skulle skje i distriktene, ved etablering av to- og treårige yrkesrettede utdanninger. Dette ble fulgt opp av regjeringen og Stortinget. DH-konseptet hadde gitt dataundervisning en sentral plass, men hovedsakelig som redskapsfag innen studieretninger for økonomi og administrasjon. Det var kun ved Molde, Østfold og Agder DH at det ble etablert et toårig spesialstudium i EDB. På denne tiden pågikk også en kraftig opprustning av den lavere og midlere tekniske utdanningen, og den toårige ingeniørhøgskolen ble normen. Mange steder gikk ingeniørutdanningen inn som en del av distriktshøgskolene. I tråd med Ottosen-komitéens innstillinger ble ressursene rettet inn mot denne storstilte satsningen på en kortere desentralisert utdanning.

Tilstrømmingen til Det matematisk-naturvitenskapelige fakultet i begynnelsen av 70-årene relativt moderat, særlig sammenlignet med resten av universitetet, som opplevde en studentboom. Interessent blant studentene for fakultetets datatilbud aksellererte stødig i løpet av tiåret som ledet opp til instituttstiftelsen, noe som skapte store problemer for flere institutter. Særlig Matematisk institutt, Avdeling D, og linjene for kybernetikk og elektronikk ved Fysisk institutt ble berørt: når et relativt marginalt område ved fakultetet, som datafagene i realiteten var, fikk en så stor etterspørsel, ble det raskt en sterk ubalanse i undervisnings- og veiledningsbelastningen. 

De ansatte ved de andre avdelingene ved Matematisk institutt og de fleste andre instituttene ved fakultetet hadde relativt rolige tider, men det kunne ikke sies for deres kolleger ved Avdeling D. De fikk hendene så fulle med utarbeidelse av undervisningsmateriell, undervisning og veiledning at det ble så godt som umulig å få tid til forskningsrelaterte aktiviteter. Dette gjaldt særlig databehandlerne, og de som i første rekke måtte bære lasset i begynnelsen av 70-årene var først og fremst professor Ole-Johan Dahl, sammen med universitetslektorene Arne Jonassen og Olav Dahl. Ved Fysisk institutt var det lignende forhold; studentinteressen for kybernetikk var stor. Instituttet befant seg på slutten av 60-tallet plutselig i en situasjon der en stor del av studentene ønsket hovedfag i en fagretning hvor det ikke fantes undervisningstilbud!\footnote{For en nærmere beskrivelse av dette henvises til artikkelen om Cybernetisk Selskaps fødsel.} Stillingene som ble opprettet for Lars Walløe, Ellen Hisdal og Rolf Bjerknes kom som et svar på dette presset.\footnote{Elektronikklinjen  var  også  utsatt,  men  ikke  i  samme  grad  som  kybernetikkmiljøet.} 

Hos Avdeling D ble det satt i gang en utreding som alle de tilsatte utarbeidet, og formulert av avdelingsbestyreren, universitetslektor Arne Jonassen. Denne fikk det beskrivende navnet 'Gjøkungen', og ble et godt dokumentert nødsskrik som malte et tydelig bilde av dramatikken i situasjonen.\footnote{Navnet på utredningen kom fra Arne Jonassen, som mente gruppen kunne «bli en gjøkunge på Matematisk institutt og spise resten av instituttet ut av redet» (hentet fra \textit{Institutt for informatikk (UiO)} på Wikipedia).} Utredningen konkluderte ikke sterkere enn at fakultetet i nær framtid burde vurdere organiseringen av informatikkens administrative plassering på lang sikt. På tross av dette vil det ikke være urimelig å anse 9. mars 1974, datoen utredningen ble lagt fram, som unnfangelsesøyeblikket for instituttet. Vi kan også se fra utredningen at begrepet 'informatikk' ble brukt som samlebegrep for den datarelaterte undervisningen ved fakultetet.\footnote{I følge utredningen var dette i tråd med hva som var vanlig internasjonalt, særlig i Europa.}

'Gjøkungen' resulterte i at fakultetet satte ned en komité, kalt Informatikk-komitéen, «for å vurdere datafagenes ressursmessige stilling og administrative plassering ved fakultetet».\footnote{\textit{Datahistorien ved Universitetet i Oslo: Fra begynnelsen til ut i syttiårene.} (1996)} Innstillingen fra Informatikk-komitéen kom i juni 1975, og konkluderte enstemmig med at det burde opprettes et nytt institutt bestående av «numerisk matematikk, databehandling, kybernetikk og digitalteknikk». Derimot så ikke komitéen noe behov for 'administrativ databehandling', som de mente var dekket andre steder, bl.a. i Bergen ved Handelshøyskolen og Institutt for informasjonsvitenskap. Informatikk-komitéen ble fulgt opp av utredninger om geografisk samling, samt forslag til ny studieplan, og i desember 1975 kunne fakultetet fatte vedtak om instituttstiftelsen med virkning fra 1. januar 1977.

Fra starten av satset det nye instituttet altså på numerisk matematikk, databehandling og kybernetikk, med databehandling og kybernetikk som de mest populære feltene blant studentene. Fra 1. april 1977 ble Kristen Nygaard ansatt som professor II, og den undervisningen og forskningen som senere ble konstituerende for den populære faggruppen for systemarbeid ble organisert rundt ham. I 1980 ble Yngvar Lundh, som hadde hovedstillilng ved FFI, tilsatt som professor II, og dette markerte starten på den organiserte undervisningen i digitalteknikk. Tre år senere ble Fritz Albregtsen tilsatt i en NAVF-finansiert laboratorieingeniørstilling. Med det var også Bildebehandlingslaboriatoriet etablert. Dette ble grunnstrukturen ved instituttet de neste ti årene.

\section{Stillingsressursene}

Ressurssituasjonen i denne 'svangerskapstiden', såvel som i tiden etter instituttfødselen, var fortsatt mager. Informatikk-komitéen hadde konkludert med at et institutt ville ha behov for 29 vitenskapelige stillinger (inklusive fem II-stillinger) og tre administrative stillinger. Instituttets behov for teknisk assistanse ble det antatt at kunne dekkes av EDB-senteret, samt av to rekrutteringsstillinger (vitenskapelige assistenter). Den faktiske situasjonen var imidlertid at miljøene som var aktuelle i instituttet bare disponerte 17 vitenskapelige stillinger (inklusive to II-stillinger), én kontorstilling og ingen tekniske vitenskapelig assistentstillinger. 

Selv om alle virket sympatisk innstilt til det nye instituttet var det altså et stort gap mellom det behovet som ble anslått og de stillinger som var tilgjengelig. Øremerkede ressurser over statsbudsjettet forekom nesten ikke. Det var stillingsstopp til UiO. De stillinger som ble tilført det nye instituttet var derfor kun de stillinger som var fylt av de vitenskapelig ansatte fra Matematisk institutt (Avdeling D ble flyttet i sin helhet), samt kybernetikkgruppen fra Fysisk institutt.

Omtrent alle ressurser til det nye instituttet måtte altså hentes gjennom intern omrokkering av fakultetets \textit{eksisterende} ressurser, som aldri er en lett prosess. Fakultetets dekanus, Tore Olsen, var imidlertid svært innstilt på at prosessen skulle lykkes. Som professor i elektronikk og tidligere bestyrer ved Fysisk institutt hadde han førstehåndskjennskap problemstillingene, og fikk Fysisk institutt både til å avgi ressurser og sin fagretning for kybernetikk. Mikroelektronikkmiljøet ved elektronikklinjen, derimot, ble beholdt ved Fysisk institutt, selv om det ble understreket at digitalteknikk var et naturlig interessefelt for det nye instituttet. Et særegent problem var de ikke-vitenskaplige stillingene. Et eget institutt forutsatte egen administrasjon og egen teknisk stab. Administrasjonen besto fra starten av en kontorstilling som ble overført sammen med Avdeling D, og av instituttsekretær Elisabeth Hurlen som ble nyansatt i halv stilling.

En annen årsak til at det nye instituttet ikke fikk tilført flere stillinger var at det i årene rundt instituttstiftelsen var tegn til at studenttilstrømningen ville flate ut. Mange dro derfor raskt den konklusjonen at interessen for data i ungdomsmassen hadde kuliminert, men dette viste seg å være en sterkt forhastet konklusjon. Studentpresset økte raskt til nye høyder. Instituttet styrket stadig sin stilling som MatNat-instituttet med det suverént verste tallmessige forholdet mellom lærere og studenter. Selv om instituttet som nevnt møtte en betydelig velvilje i fakultetsledelsen, var det likevel begrenset hva fakultetet kunne bidra med. På tross av voksesmertene øket tallet på ansatte jevnt og trutt. Ti år etter instituttstiftelsen hadde instituttet kommet opp i 49.5 stillinger, dvs. en økning på 30 siden starten. Over halvparten av disse stillingene var blitt tilført via omdisponering på fakultetet.

I 1979 ble det vedtatt et 'Program for styrking av fagområdet informatikk' der man gikk inn for en fordobling av instituttets utdanningskapasitet. Programmets målsetting ble oppnådd, både med hensyn til antall nye stillinger og utdanningskapasitet, men noen bedring i arbeidsforholdene for de ansatte ble det dessverre ikke. Fakultetet vedtok et nytt program høsten 1984, 'Program for videre utbygging av fagområdet informatikk', hvor målsettingen eksplisitt siktet på en fordobling av antall ansatte ved instituttet. På grunn av knapphet, både hva angikk stillingsressurser og antatt antall kvalifiserte søkere, ble det hevdet at det ikke var realistisk å klare mer enn halvparten av denne fordoblingen innen 1990. Det så således ikke lyst ut for en rask forbedring av arbeidsforholdene.

\section{Utstyr}

Tekniske stillinger ble ikke ansett som nødvendig for det nye instituttet. UiOs EDB-senter hadde hele tiden stått for maskinutrustningen, både til studenter og forskere. Ressurssituasjonen var heller ikke slik at det kunne være på tale å bygge opp en egen maskinpark for instituttet. EDB-senteret ytte i 70-årene en betydelig bistand, både teknisk og faglig, ved å stå for mye av hovedfagsveiledningen. Tilstrømningen økte, og det ble opprettet en egen terminalstue for laveregradsstudenter i EDB-senterets regi, samtidig som kravene til EDB-senterets virksomhet fra resten av universitetet økte. På grunn av dette kunne samarbeidsklimaet til tider bli lett anspent, spesielt siden informatikkmiljøet av og til hadde følelsen av å ikke bli prioritert med sine behov. Enda kan man høre historier om hvordan hullkortbunkene til Ifi-ansatte hadde lett for å havne i gulvet på EDB-senteret dersom man ikke hadde den rette holdningen til de maskinansvarlige. Slike ekstreme hendelser var nok ikke dagligdagse, men det var nok uunngåelig at interessene til de to datamiljøene skilte lag etterhvert som kravene fra omverdenen økte.

Utviklingen av instituttets egen maskinpark og nett skjedde først fra 1980, da Tor Sverre Lande ble ansatt i en amanuensisstilling. Han hadde i løpet av den følgende perioden nærmest eneansvaret for den tekniske kompetansen. Utover på 80-tallet oppsto det spørsmål om hvilken strategisk utstyrspolitikk instituttet skulle legge seg på, og instituttet samlet seg om en politikk som bygde på distribuerte løsninger med arbeidsstasjoner og servere. Denne var basert på programvare som skulle gjøre instituttet i størst mulig grad uavhengig av enkelte maskinleverandører. Den rådende tankegangen ellers i dataverdenen var, i kontrast med instituttets, å satse på store sentrale maskiner dominert av en enkelt utstyrsleverandør. EDB-senteret var på denne tiden representant for en slik politikk, som også passet godt inn strategien til f.eks. Norsk Data. Da instituttet i 1982 ble tilkoblet Internett og tok i bruk Berkeley UNIX, visstnok først i Norge, gikk det derfor mot strømmen, men utviklingen i senere tid har vist at det var en meget framsynt plan. I dag har denne nemlig fått alminnelig oppslutning, både nasjonalt og internasjonalt.
