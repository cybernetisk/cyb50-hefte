\chapter[Navet]{Navet - før og nå}

\textit{Red.anm.: Dette kapitlet består av to tekster, en gjenbrukt tekst fra CYB sitt 42-års jubileumshefte, og en nyere tekst av Carl-Magnus Lein, nåværende leder i Navet.}

\section{Opprettelsen av Navet}

\author{Skrevet av Anna Dahl}

Navet har sitt opphav både i dagen@ifi og i bedriftskomitéen til CYB. I mange år arrangerte CYB næringslivsdag, og idéen til dagen@ifi sprang delvis ut fra et ønske om å stable noe sånt på beina igjen, i tillegg til å ta opp tradisjonen med foajéfest - også noe CYB vanligvis hadde arrangert.

I 2006 hadde IT-bransjen hentet seg opp igjen etter det store .com-sprekket, og dagen@ifi gjorde det svært godt økonomisk - næringslivets interesse for å profilere seg overfor studentene bare økte. CYB fikk en del henvendelser fra ulike bedrifter videresendt fra instituttet, og enkelte kom direkte fra bedrifter der noen husket at de hadde vært <<bedriftsmedlemmer>> av CYB noen år tilbake. Det var pinlig åpenbart at CYB, hvis bedriftskomité i praksis lå brakk, ikke hadde noe godt mottaksapparat for denne typen henvendelser. I tillegg hadde flere av foreningene sporadiske arrangementer med forskjellige bedrifter, og behovet for koordinasjon begynte å oppstå.

Daværende bedriftsansvarlig i CYB, undertegnede, kom på bakgrunn av dette frem til at en eventuell gjenoppliving av bedriftskomitéen måtte bli nok et samarbeidsprosjekt mellom foreningene. Det var forbausende lett å rekruttere oppegående foreningsmennesker til dette prosjektet. Ettersom det ville være en del penger involvert, ble det besluttet å opprette en egen forening med tilhørende statutter, konti etc. Det var også åpenbart at man ville få behov for en mekanisme som kunne sørge for at disse midlene kom samtlige Ifi-studenter til gode, og et eget utvalg - fordelingsutvalget - ble opprettet med dette formålet.

Oppstartsmøtet ble holdt 16. februar 2006, med to representanter fra CYB (Anne Marie Bekk og undertegnede), to fra Mikro (Håkon Olafsen og Håvard Pedersen) og to fra ProsIT (Christian Mikalsen og Tommy Gudmundsen). Blant pionérene var også Geir Nilsen og Magne Eimot. I ekte gründer-ånd startet man med å arbeide seg frem til et navn, en logo og en visuell profil, og fikk opp en webside. Ettersom foreningen skulle fungere som et sentralt kontaktpunkt mellom næringslivet, studentene og i noen grad instituttet, dukket begrepet <<hub>> forholdsvis raskt opp,
og dette ble oversatt til Navet. Etter noen måneder viste det seg at vi ikke hadde vært helt alene om disse tankene, da den nye etaten NAV ble lansert - men da var det allerede blitt i overkant ressurskrevende å endre det.

Med få endringer er informasjonsteksten om Navet den samme som i oppstarten:

\begin{quote}
	Navet er bedriftskontakten ved Institutt for informatikk ved Universitetet i Oslo. Hensikten med Navet er å gjøre det enkelt for bedrifter å komme i kontakt med studentene ved instituttet, ved å tilby:
	
	\begin{itemize}
		\item et sentralt kontakt- og koordineringspunkt for alle bedriftsrelaterte aktiviteter ved instituttet.
		\item praktisk hjelp ved bedriftspresentasjoner og andre typer arrangementer (romreservasjon, plakatopphenging, utsendelse av SMS mm.).
		\item oversikt over bedriftsrelaterte aktiviteter for studenter.
	\end{itemize}
	
	Engasjerte studenter tok initiativet til å starte bedriftskontakten, og Navet er studentdrevet.
\end{quote}

Høsten 2006 kom det inn nye friske krefter i tillegg -- Are Wold, Daniel Chaibi, Fredrik Klingenberg og Magnus Korvald -- og aktivitetsnivået økte. Det første store arrangementet med en enkeltbedrift var da Google kom og holdt foredrag 5. oktober. Den nyoppstartede norske avdelingen i Trondheim stilte opp med flere representanter, og det var enorm interesse: Store Auditorium ble smekkfullt, folk måtte sitte i trappene for å få plass, og en del måtte snu i døren. Sluttfakturaen for mat og drikke kom på oppunder 22 000 kroner (!).

Navet-styret hadde ikke formelle roller i starten, men da Daniel Chaibi ble valgt til leder i 2007 økte aktivitetsnivået igjen betraktelig - særlig promoteringen ble det virkelig fart på. I årene som har gått siden da har Navet utviklet seg til å bli en svært profesjonell og veldrevet forening, og oppfyller i aller høyeste grad sitt formål om å være et kontaktpunkt mellom Ifi-studentene og Næringslivet.

\section{Navet i dag}

\author{Skrevet av Carl-Magnus Lein, nåværende leder}

Navet er i dag en sterk forening bestående av 28 medlemmer, hvor 18 er interne og 10 er styremedlemmer. Det er også rundt 40 alumni tilknyttet foreningen.

Konseptet til Navet har holdt seg godt, noe som også vises med at informasjonsteksten ikke har endret seg stort siden foreningen ble opprettet for nesten 13 år siden. Det er et konsept som fungerer godt og det er noe som gjenspeiles i dagens aktivitetsnivå; det holdes  45-50 bedriftspresentasjoner i året.

I tillegg til bedriftspresentasjoner bringer også Navet stillingsannonser til studentene, og her kan man finne både fulltid og deltidsstillinger i store og små selskaper, privat og offentlig, godt etablerte og startups.
