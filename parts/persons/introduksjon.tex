\chapter*{Introduksjon}

\setcounter{footnote}{0}

Vi har valgt å ha med denne delen primært for å ta vare på historien om noen av de som har vært involvert i CYB og Ifi sin historie, men også for å hedre dem som fortsatt er blant oss.

Ole-Johan Dahl og Kristen Nygaard, kan med rette ses på som hjørnesteiner i instituttets historie. De er faglige bautasteiner på hvert sitt vis, noe tildelingen av Turing-prisen er et bevis på\footnote{En anerkjent pris innenfor informatikk, og som gjerne sammenlignes med Nobelprisen}. Et annet bevis på deres betydning for miljøet er bygningene som pryder deres navn.

Rolf Bjerknes og Elisabeth Hurlen har også vært sentrale ansatte ved instituttet. De var viktige støttespillere for mange av dem som tok sine studier eller jobbet på Ifi, noe kallenavnene <<onkel Rolf>> og <<mor Beth>> henviser til.

Det er mange andre personer som har vært involvert i CYB og på Ifi. Det er dermed ingen lett sak å velge ut hvem man skal skrive om i forbindelse med en jubileumsbok. Utover de nevnte personene valgte vi derfor å fokusere på dem som har fått CYBs æresmedlemskap, for å få frem det arbeidet og engasjementet som de har gjort for CYB og Ifi.
