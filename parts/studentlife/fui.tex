\chapter[FUI]{Fagutvalget ved Institutt for informatik - Før og nå}

\author{Skrevet av Dennis Norheim}

25. oktober 1976 holdt for første gang CYB og Fagkritisk Gruppe (FKG) ved databehandling ved matematisk institutt felles styremøte. På møtet ble det bestemt at man skulle skaffe representanter til det kommende instituttrådet, samt at man skulle opprette et fagutvalg på informatikk. Til begge organene ble det bestemt at CYB og FKG skulle skaffe hhv to og tre representanter. 

1. januar 1977 var instituttet formelt etablert, og med det var også Fagutvalget på Institutt for informatikk formelt etablert. Fagutvalgets rolle skulle være å opprette kontaktpersonordning, se på undervisningssituasjonen og ressursbehov og administrere lesesalsplasser på informatikk.

Vi i fagutvalget har siden starten jobbet med alt fra studentenes rettigheter opp mot instituttet, fakultetet og UiO, til i nyere tid være med å bestemme hvordan Ole Johan Dahls-hus skal fungere. Det blir jobbet målrettet mot et institutt og fagmiljø som kommer studentene til gode, vi har en representant i de fleste møter på instituttet og i de fleste utvalg. Her har vi i en årrekke kommet med studentenes syn og meninger på saker som omhandler studentene. 

Fagutvalget gjennomfører kursevalueringen hvert semester, da blir hver student invitert til å svare på hvordan emnene de har tatt har blitt gjennomført. Videre setter fagutvalget seg ned for å skrive et sammendrag om alle svarene de har fått, disse sammendragene blir så publisert slik at studenter som skal velge emner kan velge det riktige emnet for sin egen del. Basert på tilbakemeldingene her har fagutvalget hvert år kåret årets foreleser. 

Fagutvalget har gjennomført hvert år siden 1990 et seminar for introduksjonsfaget i programmering, dette for å hjelpe alle studenter som lærer seg programmering for første gang eller vil bli utfordret litt ekstra. Det har i de senere årene også blitt gjennomført et forkurs i informatikk, slik at alle har grunnleggende kunnskap. 

Tidligere hadde fagutvalget ansvar for lesesaler og kollokvierom, dette ansvaret har instituttet tatt over, men vi bidrar fortsatt til med dette arbeidet.
