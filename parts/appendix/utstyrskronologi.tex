\chapter*{Utstyrskronologi}

\begin{description}
	\item[1977] Første mikromaskin\slash PC: MYCRO-1. Innkjøpt av Per Ofstad for Musicus-prosjektet.
		\item Egen terminalstue for lavere grad ved EDB-senteret med 15 skjermterminaler (Behive).
	\item[1981] Første arbeidsstasjon (PERQ) med rastergrafisk skjerm.
	\item[1982]  Egen stormaskin (DEC 20) for undervisning. Administrert av EDB-senteret.
	\item Første bruker av Tandbergs moderne terminal Universitetets første VAX 11/780 til ansatte.
	\item Første Berkeley UNIX i Norge.
	\item Tilkoblet Internett.
	\item Egen terminalstue med mikromaskiner (10 stk ALTOS).
	\item[1985] Mikro VAX med x-windows (installasjon nr. 3 på verdensbasis, utenfor MIT).
	\item[1987] Europas største installasjon av distribuerte systemløsninger basert på SUN-utstyr (både for lavere grad og til arbeids- stasjoner for hovedfag/ansatte).
\end{description}
