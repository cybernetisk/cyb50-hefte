\chapter[Chris Maile]{Chris Maile -- En whiskykjenner uten like}

\author{Skrevet av Torgeir Lebesbye}

\begin{wrapfigure}{l}{6.5cm}
	\centering
	\includegraphics[width=0.35\textwidth]{images/chris-maile/profil.png}
	\label{fig:chris-maile}
	\caption{Illustrasjonsbilde av Chris Maile.}
\end{wrapfigure}

Christopher John Maile er en vaskeekte skotte som siden årtusenskiftet har reist Norge på kryss og tvers for å lære nordmenn om de gyldne, skotske dråper. Han holdt det første whisky-seminaret for Cybernetisk Selskabs medlemmer vårsemesteret 2009, og har siden holdt et seminar i semesteret.

Under seminarene serverer Chris av sin brede innsikt om whiskyens verden akkompagnert med barndomshistorier fra The Isle of Skye på Skottlands vestkyst, vitser om engelskmenn og sekkepipespill.

Chris var initiativtaker til Oslo Whiskyfestival, som han har arrangert siden 2004. I 2005 ble han utnevnt til \textit{Keeper of the Quaich}, en ærestittel for lang og tro tjeneste i den skotske whiskyindustrien. Som eneste i Norge ble han i 2016 utnevnt til den høyeste utmerkelse i industrien, \textit{Master of the Quaich}.

På CYBs generalforsamling høsten 2013 ble Chris utnevnt til æresmedlem av foreningen for det som etter ti semestre hadde blitt CYBs lengste tradisjon som fortsatt gjennomføres, og det 20. seminaret ble holdt høsten 2018 så det ser ut til å være en tradisjon som vil kunne fortsette å glede nye cybbere i lang tid.
