\chapter{IFI-Skitur}

\author{Skrevet av Arne Hassel}

Tradisjonen med skitur for studentene på Ifi startet i 2006, da med ProsIT i regien. ProsIT var interesseforeningen for profesjonsstudentene og en viktig del av oppgaven deres var å skape sosiale tilbud for sine studenter. Hemsedal-turen, som det het da, var et av årets høydepunkter. Turene var hadde høy stemning med rundt 48 studenter fordelt på 3-4 hytter. Det ble holdt quiz på veien opp, gode turer ut i bakkene på dagtid og fest og moro på hyttene på kveldstid. Bussturene tilbake til Oslo var preget av ro og fred, da alle var for slitne til å gjøre noe mer\footnote{Med andre ord har ikke innholdet i skituren forandret seg nevneverdig}.

I starten sponset Ifi hele turen, men etterhvert kuttet de støtten ned til 30 000 kr og ProsIT måtte skaffe resten av midlene selv. ProsIT begynte å ta betalt for billettene til turen, og med det tok de også en hundrelapp ekstra per deltaker. Denne hundrelappen gikk til andre sosiale arrangement foreningen holdt.

Da det ble klart at profesjonsstudiene skulle opphøre ved Ifi var det flere som lurte på hva som ville skje med skituren. Cybernetisk Selskab ble spurt om de ville ta over driften av arrangementet, men takket nei da de var redd turen, som hadde fått rykte på seg å innebære mye fyll, skulle ha negativ påvirkning på arbeidet som ble gjordt med å få i gang Escape. Det så dermed ut til at turen ville ende med sin siste tur i 2009.

Et par ildsjeler nektet derimot å se dette skje. Anders Asperheim og Nikolai Kristiansen tok initiativ for å få CYB til få til noe, og spurte om hjelp fra Martin Haugland og Arne Hassel. De to sistnevnte var godt bevandret i CYB og administrasjonen på Ifi, og visste litt om hvilke tråder man kunne trekke i for å få ting til å skje likevel. Det ble bestemt å trekke frem det sportslige i arrangementet og fokuserte på at folk skulle ut og ha det gøy i bakkene. Et grep som ble tatt var blant annet å kjøpe inn skikort for deltakerne. Dermed ble det sportslige en obligatorisk del av turen. Det gjorde også at CYB kunne få til gode pakkeavtaler.

At turen skulle gå til Hemsedal var ikke spikret, og man sendte forespørsler til Hafjell og Trysil i tillegg. CYB fikk raskt svar fra alle, og foreningen endte likevel opp med å velge Hemsedal. Der fikk CYB en komplett pakke, med buss, opphold og skikort. Budsjett ble laget og det ble søkt om støtte fra både Ifi og Fordelingsutvalget, på 20 000 kr hver. Etter litt om og men fra FU ble søknaden innvilget, og påmeldingen kunne starte for fullt. Det var 48 plasser tilgjengelig og break-even gikk på 41. Turen ble derimot helt full, og det ble feiret ved å spandere middag på alle hyttene.

Turen oppover gikk til quiz, med premiering av en kasse øl til vinnerne, og da bussen kom frem til anlegget nøyaktig 13:37 var stemningen på topp! At det var en siste runde med spørsmål inspirert av Paradise Hotel kan også ha vært en god pådriver for den gode stemningen. Leilighetene lå i bunnen av bakken, noe som gjorde at folk kom seg fort på plass, og ut i bakkene.

Nytt for turen var et par konsept. Felles bildesesjon hvor folk kunne samles og ta bilder, og Ifi-lekene på lørdag kveld hvor deltakerne samlet seg og kjørte lagkonkurranser. Det første var heller laber interesse for, men sistnevnte var en braksuksess, og da CYB ikke klarte å følge det opp året etter fikk de mye kritikk for det. Ifi-lekene har siden vært et fast innslag på turene på Hemsedal.

Siden CYB sin overtagelse i 2010 har de hvert år arrangert turen, og har nå økt antall busser til to, slik at de får hele 96 sjeler med seg. CYB har også valgt seg hyttene i Veslestølen igjen (hvor også ProsIT hadde oppholdene sine), hvor det er mer plass, og ikke minst badstue. Rundt disse har noen modige sjeler prøvd å starte en badstue-turnering, hvor de løp og snøbadet mellom alle badstuene på hyttene. Men siden dette har ikke vært så populært blant alle turdeltakerne (ikke minst blant dem som ikke setter pris på å få halvnakne, svette kropper inn i hytta si) så har ikke dette blitt en tradisjon.

I 2019 går turen inn i sin 14. gjennomføring, og denne gang er det skiftet sted. Nå er Hafjell blitt turens destinasjon, og innen denne boken er kommet ut så er turen gjennomført. Som tidligere arrangør og deltaker på flere turer håper jeg det blir en suksess, og at studentene fortsetter å storkose seg i lag med sine medstudenter, både med tanke på det sportslige og det sosiale!
