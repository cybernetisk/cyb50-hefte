\chapter[Ifi-ordenen]{Hennes Majestet Keiserpingvinen den Fornemmes orden}

\author{Skrevet av Arne Hassel}

Hennes Majestet Keiserpingvinen den Fornemmes orden, eller Ifi-ordenen som den er bedre kjent som, er en studenter-ridderorden ved Ifi som har som formål å hedre personer som har gjort en beundringsverdig innsats for studentmiljøet på Ifi. Dette er som regel studenter, men inkluderer også ansatte og andre som har gjort mye for studentmiljøet.

Studenter-ridderordener finnes det flere av, og i mange varianter. Av de større kan man nevne Store Bjørn ved Realistforeningen, Den Gyldne Gris ved Det Norske Studentersamfund og Bukkeordenen ved Blindern Studenthjem. I Trondheim finner man ridderordener som Det Sorte Faars Ridderskab ved Studentersamfundet, i Bergen finner man Pinnsvinsordenen, og på Ås finner man Hans Hovenhet Hestehoven.

Kimen til en egen ridderorden på Ifi oppsto i arbeidet til CYB42, da man skulle feire at Cybernetisk Selskab feiret 42 år. Med nytt bygg, ny studentkjeller og et voldsomt engasjement følte man at det var plass for en egen studenter-ridderorden på Ifi. Gjennom arbeidet i CYB42 tok Klaus Wik på seg arbeidet med å sette opp en plan for hvordan dette kunne gjennomføres. På CYBs Generalforsamling våren 2011 presenterte han planene, og med det ble et skyggekabinett opprettet som skulle finne de første kandidatene. Presentasjonen ble humoristisk fremført i ekte Steve Jobs-stil med svart turtleneck, og til spørsmålet om medaljer trengte å være en del av den ordenen hadde han en egen slide med tittelen <<Medaljer er kult>> sammen med bilder av Stalin, Idi Amin og Khadaffi.

Slik ble det til at på den første Ifi-gallaen, 22. oktober 2011, gjennomførte man den første seremonien hvor kandidater ble utnevnt. Geir Arild Byberg og Øyvind Bakkeli fikk æren av å bli de første til å motta ordenen med hhv Storkorset og Kommandør-medaljen. Som Stormester tok Geir Arild over styringa av ordenen, og de neste årene var det spesielt dem som hadde arbeidet mye med å bygge opp CYB og Escape som ble hedret.

Da Geir Arild ønsket å trekke seg tilbake i 2015, var det et spørsmål om hvem som skulle ta over. Arne Hassel, som hadde blitt tatt opp som Kommandør året før, stilte opp, og valgte med det å gjøre noen forandringer. Det hadde vært mange diskusjoner om ordenen skulle være til for CYB eller hele studentmiljøet. Ved å også introdusere seremoniene på Foreningsfesten på våren så ønsket man å signalisere at det skulle være for hele studentmiljøet. Med dette begynte også antallet utnevnelser å ta seg opp, med hele åtte tildelinger i løpet av 2015. Dette førte til en dobling av personer med ordens-tildelinger, og flere skulle det bli.

Det er utrolig mange flinke og oppegående mennesker på Ifi, og det har det også vært gjennom mange år. De som jobbet med ordenen følte et press på å <<ta igjen backloggen>> av mennesker som hadde vært aktive en stund tilbake. Dette, i kombinasjon med at man ønsket seg bedre representasjon av studentene, gjorde at man arbeidet aktivt for å utvide søket og arbeidet med å finne verdige kandidater. Ikke at det var et problem å finne gode kandidater som ikke var hvite menn, men det var viktig å ha det i tankene, og ta noen ekstra runder for å få gode signaler fra hele studentmiljøet. Det gjorde at man i 2016 også tok opp åtte nye medlemmer, og man begynte å se en bedre representasjon\footnote{Det jobbes fortsatt med dette, målet er at man på sikt burde ha en representasjon som er på linje med sammensetningen i studentmiljøet}. Det ble også gjennomført en ekstra seremoni på fem års feiringen av Escape, hvor man ville hedre dem som hadde jobbet mye med studentene i flere år og dem som gjorde en ekstra innsats for at Escape kom på plass og ble som den ble.

På ifi-gallaen høsten 2016 valgte man også å gjøre en omjustering av titler på de som allerede hadde mottatt en orden ifm at man introduserte flere titler; man gikk nå fra å ha Stormester, Kommandør og Ridder til å også ha Kommandør med stjerne og Ridder av første klasse. I den forbindelse fikk Suhas Govind Joshi æren av å bli ordenens første Kommandør med stjerne, til stor jubel fra forsamlingen.

De neste par årene har man fortsatt å ta opp mange kandidater, med hhv seks kandidater i 2017 og fire kandidater i 2018. Spesielt minneverdig er utmerkelsen av Nikolas Papaioannou som Kommandør, hvor det ble både trampeklapp og et og annet vått øye.

På Ifi-gallaen i 2018 valgte man for første gang å ikke ta opp kandidater, rett og slett fordi man ønsket å vente til Jubileumsgallaen i februar neste år, og med det gi personene som man mener fortjener en hedersbetegnelse en ekstra stor ære.

2019 er duket til å bli et stort år, med 50-års feiring av Cybernetisk Selskab og en fantastisk Jubileumsgalla! Hennes Majestet Keiserpingvinen den Fornemme gleder seg til lørdag 16. Februar, til en forrykende kveld med glede og moro, og ikke minst æren av å innlemme enda flere verdige kandidater i sin orden!

Studentmiljøet på Ifi er utrolig godt, og det er viktig å sette pris på det samholdet og arbeidet som gjøres. Keiserinnen og de som jobber med hennes orden håper de kan bidra til dette, og setter pris på all hjelp man kan få. Sjekk gjerne ut ordenen.ifi.uio.no, hvor man også kan nominere personer man føler fortjener en ekstra heder.
