\chapter{Fadderstyret ved Institutt for informatikk}

\author{Skrevet av Thao Tran, leder for Fadderstyret ved Institutt for informatikk, og Arne Hassel}

Fadderordningen på Ifi har alltid vært nært tilknyttet studiemiljøet og ildsjelene som ``var med på alt''. Gjennom historien har fadderordningen vært gjennomført i regi av alt fra grupper uten nær tilknytning til noen foreninger, til grupper med nær til èn bestemt forening, til grupper med tilknytninger til flere foreninger. Det var også tendenser til mer oppsplittede fadderordninger, hvor bestemte grupperinger gjennomførte opplegg for ``sine'' studenter og ikke nødvendigvis hadde fokus på alle som skulle starte på Ifi.

I 2009 ble Cybernetisk Selskab spurt om å ta en aktiv rolle i fadderordningen, i håp om å skape rammer for en fadderordning som inkluderte alle, uavhengig av hvilken studieretningen man startet på. Denne oppfordringen takket man ja til, og frem til med 2011 tok foreningen ansvar for å samle et fadderstyre som så tok kontroll på egen hånd videre. I 2012 kuttet fadderstyret de organisatorisk båndene til CYB, og i 2014 ble Fadderstyret ved Institutt for informatikk opprettet som egen forening, med egne vedtekter og gjennomføring av generalforsamlinger.

Fadderstyret har ansvar for å ta i mot de nye studentene de to første ukene på høstsemesteret, og bidrar også til opplegg for dem som starter på vårsemesteret. Arbeidet er todelt, først og fremst få rekrutert engasjerte studenter til å være faddere, men også planlegge programmet. Programmet varierer litt fra år til år, men høydepunktene er gjerne gaffateip-, CTRL+ALT+DEL- og Pub til pub-festen\footnote{Kjærlig kalt P2P-festen av enkelte informatikere}. Men det arrangeres også alkoholfri arrangement, som Ifi-olympiaden og spillkvelder for både gaming-, brettspill- og sjakk-interesserte.

Studiemiljøet er hjørnesteinen av fadderstyret, og hovedfokuset er å skape et miljø som får studenter presentert for studieplassen på best mulig måte. Ved å fremme at Ifi er et multikulturelt sted som dekker hver enkelt students behov, bidrar det til å skape en trygghet på studiet, men også bedre den sosiale fronten.

I flere år har fadderstyret stått for en fadderuke, som i år blir omdøpt til Studiestart uken, hvor hovedfokuset er å få nye studenter til å trives på de ulike programlinjene. Ved hjelp av motiverte faddere ved Ifi samarbeider Fadderstyret for å skape en langvarig trivsel som kan bidra til et varmt og åpent miljø hvor alle føler seg komfortable og velkomne.
