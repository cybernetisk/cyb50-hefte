\chapter{Ifi-galla}

\author{Skrevet av Arne Hassel}

Da Cybernetisk Selskab feiret 42 år var det mange tanker og ønsker som lå i lufta. Året var 2011, Ole-Johan Dahls hus hadde blitt offisielt åpnet, og med det var også åpningen av Escape endelig et faktum. CYB hadde gått fra å være en forening på kanten til å legges ned, til å få et enormt fokus på arrangement og ikke minst arbeidet med å få alt på plass til den nye studentkjelleren. Nytt bygg med masse plass, eget sosialt lokale for studentene, god rekruttering av nye interne - alt lå til rette for nye tradisjoner og festlige påfunn.

Rundt dette arbeidet ble CYB42 til, en egen festkomité som jobbet med å få til alt man ønsket å gjennomføre på dette jubileumsåret med det viktige tallet - tallet som er svaret på det spørsmålet om livet, universet, og alt mulig. Komitéen hadde planer om flere ting, som publikasjon av et nytt jubileumshefte og fikse artige ting til feiringen. Men den viktigste tingen de tok seg i fore var å gjennomføre Ifi sin første studentergalla!

CYB42 besto av flere medlemmer, men det var spesielt Iver Stubdal, Ole Kristian Hustad, Eirik Munthe, Eivind Hauger og Øystein Røysland Sørlie som jobbet mye med planlegging og gjennomføring av den første gallaen. Nå som galla har blitt en tradisjon så har man ganske god oversikt over hva som skal gjøres, men det var mange ukjente faktorer den første gangen. Hvilken dato skulle man velge? Hvor skulle middagen gjennomføres? Hvordan skulle maten fikses? Hvordan setter man opp bordplassering? Disse, og mange andre spørsmål, fant komitéen svar på, og skapte med det en ramme for senere gjennomføringer.

Kanskje den viktigste beslutningen var å gjennomføre gallaen på lørdagen i uke 42, en tradisjon CYB har holdt på siden. Nå blir det riktignok et unntak med Jubileumsgallaen, men slikt må man kunne unne seg når CYB feirer 50 år! En annen prestasjon man klarte det første året var å gjennomføre middagen i Informatikksalen, eller det flere med glimt i øyet kaller Faculty Club (sier noe om ambisjonene flere på instituttet hadde - og kanskje har enda?). Med tanke på hvor stor gallaen har vokst seg til, med over 100 gjester, så er dette ikke en mulighet lenger (eller noe CYB har fått lov til, for den saks skyld - det gikk litt heftig for seg lørdagen 22. Oktober 2011).

CYB42 fortsatte å arrangere galla et par år etter sin første gjennomføring, med en glidende overgang til 2014, da man opprettet egen komité gjennom arrangementsgruppa, og hvor et par CYB42-ere fortsatte som rådgivere.

Et nytt tilskudd som har dukket opp i senere år er presseveggen, hvor deltakerne kan ta bilder av seg selv sammen med sine venner og andre gjester. Denne blir flittig brukt utover kvelden, både med påkledde og en sjelden gang avkledde deltakere!

Noe som også har blitt et fast innslag i løpet av gallaen er seremoniene til Hennes Majestet Keiserpingvinen den Fornemmes orden, mer populært kalt Ifi-ordenen, hvor kandidater blir slått til ridder eller kommandør, ærespriser som gis på grunnlag av at personene har gjort en beundringsverdig innsats for studentmiljøet. Det var da også på den første gallaen i 2011 at Ifi-ordenen ble til, med Klaus Wik i føringen av skyggekabinettet. (Mer om dette i kapitlet om Ifi-ordenen.)

Slik galla har blitt en tradisjon, så har galla-nachspiel i Escape også blitt en tradisjon. Da er deltagerne gjerne i veldig godt humør, og det blir mer drikke og godt sosialt lag som fortsetter ut til de sene nattetimer - eller ihvertfall til halv to, når lokalet må stenge. Men innen da er det gjerne noen som har funnet ut at de vil utfordre hvor mange kan få inn i en stakkars studentbolig, og moroa fortsetter dit. Hvordan disse hyblene klarer å skaffe til veie drikke til sine tørste gjester forblir et mysterium, men det ryktes at god koordinering (og iblant ukoordinering) er trikset.
