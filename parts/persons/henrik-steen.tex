\chapter[Henrik Steen]{Henrik Steen -- 34 timer i et døgn}

\author{Skrevet av Arne Hassel}

\begin{wrapfigure}{l}{7cm}
	\vspace{-60pt}
	\centering
	\includegraphics[width=0.45\textwidth]{images/henrik-steen/profil.png}
	\label{fig:morten-daehlen}
	\caption{Illustrasjonsbilde av Henrik Steen.}
\end{wrapfigure}

Henrik er et drivjern med øye for å strukturere og effektivisere systemer. Han startet studiene sine på Ifi i 2010 og ble raskt engasjert i livet på Blindern Studenterhjem. Her var han blant annet økonomisjef og webmaster for UKA på Blindern, hvor han satte opp et billettsystem som brukes den dag i dag.

Da Henrik ble valgt inn som Kasserer i CYB høsten 2013 startet han et arbeid som blant annet resulterte i at man skiftet regnskapssystem og fikk på plass mer effektive rutiner. Internsystemet i CYB er også noe han har lagt ned en imponerende mengde timer i. Det begynte riktignok som et system for varebeholdning, men har utviklet seg til å bli et system CYB nyter godt av den dag i dag.

Våren 2017 gjorde han seg ferdig med mastergraden på Ifi, og på generalforsamling samme semester ble han tildelt CYBs æresmedlemskap. Begrunnelsen var hans enorme engasjement i foreningen, med en imponerende produksjonsevne, men det ble også bemerket hvor villig han har vært til å være med og hjelpe til i etterkant av sine engasjement også.

I dag jobber Henrik som konsulent i Capra Consulting, en jobb hvor hans produksjonsevne og engasjement får lov til å utfolde seg. Men fortsatt så hjelper han til i It-gruppa, og iblant så dukker han også opp på Ifi for å være med i arbeidet på kosetirsdager og lignende.
