\documentclass[../main.tex]{subfiles}

\begin{document}
av Birgitte Kvarme

Mye har skjedd siden 17. februar 1969, da Cybernetisk Selskab ble stiftet. Informasjonsteknologien har utviklet seg enormt, mer enn noen kunne ha ant. Cyb har selvfølgelig også utviklet seg, dog ikke så mye som faget, og har etter hvert forandret seg fra å være en liten forening med forankring i kybernetikk, til en relativt stor forening med 400-500 medlemmer hvert semester. I dag favner Cyb over alle studieretningene ved Institutt for informatikk, og er opptatt av å skape et aktivt faglig og sosialt studentmiljø ved instituttet. Det faglige kommer inn i form av foredrag og ekskursjoner, det sosiale i form av fester, filmkvelder og nachspiel. Rollen som pressgruppe og pådriverorganisasjon for studentene som vi så tendensen til i begynnelsen av Cybs historie er utvisket. I dag er det fagutvalget ved institutt for Informatikk som ivaretar studentenes interesser.

\subsection{Cybs organisering}
Styret i CYB består av 7 personer. (I 1994 har vi en prøveordning med 8 personer i styret). Hvert styremedlem blir valgt for 1 år og de fleste gir seg etter dette. Omtrent halvparten av styremedlemmene blir byttet ut hvert semester. Dette fører til en ganske stor gjennomstrømning av styremedlemmer. Dette er positivt fordi man stadig vekk får inn personer med nye ideer, men det fører også til at mye går i glemmeboken. Vi har også et fondstyre som består av 3 personer. Dette er som regel tidligere styremedlemmer.

\subsubsection{Tradisjoner}
I en alder av 25 år har vårt kjære Selskab rukket å få en del tradisjoner. Noen av disse stammer fra de første årene etter stiftelsen, andre er kommet til i nyere tid. Tradisjonene kan også forandre seg. Noen ting faller bort mens nye ting kommer til.

%TODO antar at de følgende subsubsectionene er disse ment å være underpunkter til tradisjoner-subsubsectionen?

\subsubsection{Ekstraordinær generalforsamling}
Dette har faktisk blitt en tradisjon slik at ordet "ekstraordinær" vel kan sies å være litt missvisende. Denne holdes i begynnelsen av hvert semester, og her blir regnskapet fra forrige semester lagt fram. Etter ekstraordinær generalforsamling holdes det et foredrag, og det hele avsluttes med nachspiel. Hvert semester får et av de avtroppende styremedlemmene i oppgave å lage et kryssord. Vinneren blir kåret på nachspielet, og hedret med en flaske vin.

\subsubsection{Generalforsamlingen}
Denne arrangeres i slutten av hvert semester, og her velges neste semesters styre. Etter generalforsamlingen er det nachspiel med gratis spekemat, øl og akevitt til medlemmene. Som nevnt tidligere er dette en tradisjon som stammer fra høsten 1971. Under generalforsamlingen tar vi fram våre kjære sanghefter, og synger til langt ut i de små timer.

\subsubsection{Rekeaften}
Rekeaften arrangeres hver vår. Av navnet forstår man at menyen er bestemt på forhånd. Før rekene skrelles nyder alle deltagerene en tegne- eller eventyrfilm.

\subsubsection{Servomøtet}
Annenhvert år arrangerer Norsk Forening for Automasjon sitt "Servomøte" i Trondheim samtidig med "Studentuka". Dette er vel den eneste aktiviteten der Cybs tidligere tilknytning til kybernetikk fremdeles er synlig. Turen går over 4-5 dager, der to av dagene blir brukt på Servomøtet. Servomøtet avsluttes med middag på Studentersamfundet med påfølgende fest og revy. Man har også vært på omvisninger på NTH, og besøkt studentmiljøene der. (Når en Oslo student kommer til Tronheim og får oppleve studentmiljøet der, kan man bli ganske misunnelig.)

\subsubsection{IN-fest}
IN-festen er en fest for alle informatikkstudentene. Den holdes vanligvis i Realistforeningens kjeller i Vilhelm Bjerknes hus, og det er vel ingen av Cybs arrangementer som er mer populære. Festen går vanligvis av stablen en gang i året, med Cyberlympics i diskettkasting (5 1/4 " disketter), "IFI-GOGO BAR" og mye mer.

\subsection{Drømmen om vår egen «CYB-kjeller»}
Hva er Cyb om 5 år, 10 år eller 25 år? Det er det heldigvis(!) ikke mulig å forutsi, men man kan jo gjøre seg opp noen tanker om hva man ønsker at Cyb skal være. Det man i første rekke drømmer om, er å få sin egen "Cyb-kjeller". Institutt for informatikk ligger i periferien av universitetsomerådet, litt "bortgjemt" i forhold til de andre instituttene ved Mat. Nat. Derfor er det ikke så veldig mange andre enn informatikkstudentene som går den lange veien til Ifi. Man kunne tro at en slik "isolering" ga opphav til et intimt studentmiljø, men slik er det dessverre ikke. Vi tror at et lokale der studenter kan møtes til både faglige og sosiale sammenkomser vil være en god ide. Dette er en sak som det forhåpentligvis vil arbeides med framover.
\end{document}