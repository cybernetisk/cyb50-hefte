\chapter[Ole-Johan Dahl]{Ole-Johan Dahl – Ifis faglige grunnlegger}

\author{Skrevet av Narve Trædal}

\begin{wrapfigure}{r}{6cm}
	\centering
	\includegraphics[width=0.35\textwidth]{images/ole-johan-dahl/profil.png}
	\label{fig:ole-johan-dahl}
	\caption{Illustrasjonsbilde av Ole-Johan Dahl.}
\end{wrapfigure}

\begin{figure}[h!]
	\includegraphics[width=\textwidth]{images/ole-johan-dahl/Ole-Johan.jpg}
	\label{fig:ole-johan}
	\caption{UiOs første databehandlings-professor i 1972.}
\end{figure}

Ole-Johan Dahl ble professor i numerisk analyse ved avdeling D ved Matematisk institutt i 1968. I 1970 ble stillingen omdefinert til professorat i databehandling, samtidig med at han la fram en samlet undervisningspakke for første delen (20-vekttallsgruppen) av studiet. Han var den eneste professoren i databehandlingsfaget i ti år framover. Han og hans elever og kolleger utviklet et studietilbud som var så gjennomarbeidet og banebrytende at det med små justeringer ble brukt til langt opp i 1980-årene. Det var helt på høyde med den undervisningen som ble tilbudt ved de fremste amerikanske universitetene. Selv arbeidet han i disse årene særlig med formelle strukturer i dataprogrammer, og bidrog på vesentlig vis til videreutvikling av programmeringsteori.

Ved Institutt for informatikk var Ole-Johan Dahl først og fremst interessert i programvarens pålitelighet. Han forsket innen fagfeltene programmeringsspråk, programspesifikasjon og verifikasjon, såkalte formelle metoder. Målet var å kunne resonnere om programmer etter matematiske prinsipper, slik at man kan bevise at ønskede egenskaper er oppfylt – selv før programmene kjøres. I motsetning til konkret testing av programmer, kunne man analysere programmers oppførsel for en uendelighet av ulike <<input>>, evt. alle mulige omgivelser. Dette fagfeltet belyser fundamentale vitenskapelige aspekter innen informatikk, og har påvirket beslektede felt, ikke minst språk og metodikk for programmering og problemspesifikasjon.

Selv om bruk av formelle metoder kan være tidkrevende og kostbart i praktisk programutvikling, vil formelle metoder kunne ha stor nytteverdi for anvendelser der store verdier eller menneskeliv kan gå tapt om det er feil i programvaren (eksempelvis i en autopilot). I motsetning til mange andre forskere i fagfeltet, la Dahl i sin forskning sterk vekt på databehandlingens praktiske anvendelse. Han sa ofte <<Jeg er ingen matematiker>> og unnlot å ta del i forskning som kun hadde matematisk interesse. Dahl søkte alltid etter enkelhet og eleganse og kunne være nådeløs i sin kritikk, både av seg selv og andre, når det var på sin plass, men med sin evne til å inspirere og få alle til å yte sitt aller ypperste var han samtidig en høyt elsket lærer, veileder og kollega. Han stilte store krav til den som skulle veiledes om at vedkommende på selvstendig vis skulle lage problemstillinger og fullføre prosjekter. Han var skeptisk til <<underveispublisering>> i forskningen, og mente at man burde vente til man hadde et fullgodt svar på en velformulert problemstilling, før man publiserte. Selv om han var fullt klar over at informatikken var en vitenskap rettet mot anvendelser, så sto han hardt på at studiene ved Ifi, ikke skulle være et ingeniørstudium. Sammen med Kristen Nygaard sto han klart på studentenes side, da det på slutten av nittitallet var krefter ved Ifi som ville flytte instuttet til Fornebu, og etablere et mer ingeniørpreget profesjonsstudium der ute.

Ole-Johan Dahl kom i kontakt med datamaskiner da han 1952 avtjente sin verneplikt ved Forsvarets Forskningsinstitutt (FFI). Her hadde Jan V. Garwick i 1947 fått ansvaret for matematisk analyse og beregninger, og Dahl ble plassert på <<regnekontoret>> som var ledet av Garwicks assistent, Kristen Nygaard. I årene som fulgte, utviklet FFI et faglig samarbeid med miljøet rundt universitetet i Manchester og elektronikkfirmaet Ferranti, og 1957 fikk FFI det første eksemplar av Ferrantis Mercury-maskin. Maskinen ble kalt FREDERIC, og 1958 avla Dahl som den første i Norge embetseksamen med hovedoppgave i programmering. Hans neste prosjekt var å utvikle og implementere et <<høy-nivå>>-språk for numeriske beregninger, MAC. Dahl og Garwick dannet den første viktige programmeringsgruppen i Norge. En kan trygt si at Garwick var norsk informatikks tidligste far og Dahl hans fremste elev.

Kristen Nygaard ble ansatt ved Norsk Regnesentral i 1960, og hentet i 1962 inn Ole-Johan, som hadde fortsatt som forsker ved FFI. I de kommende seks årene utviklet de to sammen det verdensberømte programmeringsspråket SIMULA, som ble grunnleggende i store deler av dataundervisning og forskning ved de fremste universiteter i mange år etterpå, og er modell for mange moderne dataspråk. Den første versjonen, med objekt- og klassebegrepene i sin opprinnelige form, var klar våren 1965. Under arbeidet med å forbedre Simula oppfant de i 1967 det som i dag er kjent som <<nedarving>> (inheritance) i programmeringspråk. Dette ledet til en grunnleggende nybearbeiding av språket til det som senere har fått navnet Simula 67, og som umiddelbart vakte stor oppsikt i fagkretser i hele verden. Idéene i Simula har blitt en modell for mange moderne dataspråk, og ble i tiårene etter den dominerende tenkemåten innenfor informatikk over hele verden. For dette arbeidet ble de to i 2001 hedret med IEEE John von Neumann-medaljen og i 2002, like før de begge døde, med mindre enn tre måneder, mellomrom, fikk de ACM A.M.Turing-prisen. Den siste blir ansett som det nærmeste man kan komme en nobelpris i informatikk.

\begin{figure}
	\includegraphics[width=\textwidth]{images/ole-johan-dahl/ieee.png}
	\label{fig:ieee}
	\caption{I 2017 ble den foreløpig siste æresbevisningen avduket i foajéen i Ole-Johan Dahls hus, i forbindelse med feiringen av Simula-språkets 50-års-jubileum.}
\end{figure}

I 2017 ble den foreløpig siste æresbevisningen avduket i foajéen i Ole-Johan Dahls hus, i forbindelse med feiringen av Simula-språkets 50-årsjubileum.

Ole-Johan var, i motsetning til sin samarbeidspartner Kristen, stille og relativt beskjeden av natur. Men han kunne være sterkt engasjert når det var noe som betydde mye for ham. Det fortelles at en nyansatt ved Norsk Regnesentral en gang midt på 60-tallet kom løpende ned til sentralborddamen og ropte: <<Hva skal vi gjøre! Det står to menn og slåss foran tavlen i 2.etasje.>> Sentralborddamen stakk hodet ut av luken, lyttet litt og sa: <<Det er ikke så farlig. Det er bare Kristen og Ole-Johan som diskuterer Simula.>>

Lidenskapen hans, i tillegg til informatikk-aktivitetene, var klassisk musikk. Han var en dyktig pianist, og var i mange år engasjert i Oslo Kvartett-forening. Han bidro til mange musikkstunder ved instituttet. Musikkinteressen hans delte han blant annet med den kjente amerikanske informatikkprofessoren Donald Knuth, som var på to forskningsopphold ved Ifi, og som ble utnevnt til æresdoktor ved UiO i 2002.
Det er ikke mange bygninger ved UiO som bærer sitt navn mer med rette enn Ole-Johan Dahls hus.

\section*{Kilder}

Du kan lese mer om Ole-Johan og Kristen bl.a. her: \url{https://www.ub.uio.no/fag/informatikk-matematikk/informatikk/faglig/dns/}
