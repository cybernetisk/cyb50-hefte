\chapter[Fra dvale til kjeller]{2000-tallet: Fra dvale til kjeller}

\author{Skrevet av Geir Arild Byberg}

I perioden 2006-2007 samlet flere nye og eksisterende studentforeninger seg til fellesmøte om fremtiden til Cybernetisk Selskab. Det var et ønske fra det daværende styret, og ledelsen på Ifi, om å få mer fart i aktiviteten til foreningen, for å unngå oppløsning og påfølgende gravøl, som selvsagt ville blitt holdt for å sikre en verdig avslutning. På møtet stilte representanter fra Fagutvalget (FUI), Mikro, ProsIT, Navet og PING for å diskutere statusen, veien videre, hvem som eventuelt skulle styre skuta, den økonomisk situasjonen, etc. Etter mye om og men ble Geir Arild Byberg valgt til ny styreleder, og Heidi-Christin Bernhoft-Jacobsen til nestleder, og hver av de andre foreningene på Ifi stilte med styreleder\slash nestleder som skulle fungere som styremedlemmer i CYB.

Likevel tok aktiviteten seg dessverre ikke noe særlig opp i de følgende månedene. Det var vanskelig å fokusere på CYB for styret, med tanke på de andre studentforeningene som ikke kunne ofres av sine respektive styremedlemmer for å bevare den ``nye'' foreningen. Etter en tid ble Geir Arild kontaktet av tidligere styremedlemmer, Ole Kristian Hustad og Martin Lilleeng Sætra, som lurte på om det skjedde noe i CYB eller om skuta var virkelig på vei ned. Dette var motivasjonen som trengtes, og det ble blåst liv i den spede troen på at CYB faktisk kunne bli noe. Det ble så gjennomført en ny (og skikkelig) generalforsamling, og det nye styret ble bestående av Geir Arild Byberg som styreleder, Øyvind Bakkeli som nestleder, Eirik Hjelle som webmaster, og med Ole Kristian Hustad, Anna Dahl og Martin Lilleeng Sætra i rollen som \textit{gammel pamp} nummer 1, 2 og 3, respektivt. Mens Geir Arild, Øyvind og Eirik bygget opp en ny webside til CYB, arrangerte den eldre garde CYB sin 35-årsdag i kjelleren på Café Abel, og så var foreningen nok engang på glid.

Nå som CYB hadde dannet et skikkelig styre og kunne man fokusere på å dra i gang aktiviteter for foreningen. ``Drit i hvor mange som kommer, la oss heller gjennomføre det!'' ble mottoet. Websiden ble et godt brukt forum, og interessen for CYB begynte å ta seg opp blant studentene. Dette var nok en resultat av at for første gang på lenge hadde Ifi en studentforening som var helt og holdent sosialt rettet, og som var for alle studentene, uavhengig om du var bachelor-, master-, profesjons- eller enkeltemnestudent.

I 2009, altså ikke så lenge etter at CYB våknet til liv, kom det en forespørsel fra styret i Realistforeningen om vi var interessert i å arrangere noe informatikkrettet i april. Dette var året da The Pirate Bay måtte bevise sin uskyld, og april hadde blitt valgt ut til å være informatikkmåneden med faglige foredrag og sosial moro. CYB stilte så opp med å arrangere et faglig seminar med navnet That’s IT!, men dessverre var den planlagte datoen for arrangementet i forhold til eventuelle andre helligdager noe dårlig gjennomtenkt, og oppmøtet 30. april var derfor noe mangelfullt. For de som møtte opp ble det derimot kake, avsindige mengder pizza, og øl, og alt i alt hadde alle en hyggelig opplevelse. Ryktene tilsier at enkelte ønsker de fortsatt var der.

I løpet av det året vokste CYB videre og arrangerte tidenes første Ping Pong turnering, tidenes første whiskysmaking, julelodding, juleprogrammering, bidro til fadderukene sammen med de andre studentforeningene og mye, mye annet. Det var samtidig derimot også planlegging på gang av annen liten hemmelighet: informatikkstudentenes første kjeller i det nye bygget.

Det er umulig å si når diskusjonene startet, men ledelsen i Ifi, med Morten Dæhlen, Line Valbø, Narve Trædal og Terje Knudsen i teten, sørget for at det nye bygget skulle ha studentkjeller. Noen av de andre studentforeningene ble spurt om å ta på seg driften, men grunnet lite mulighet til å følge opp ble det enighet om at CYB skulle få ansvaret for å drive kjelleren. Så, i et siste dytt for å få CYB ut av dvalen, ble Magnus Johansen CYBs første Kjellermogul og sørget, sammen med sin trofaste gjeng i Kjellerstyret, for at nødvendige avtaler ble inngått og personell (les: frivillige studenter) rekruttert og ivaretatt, og drømmen om en egen studentkjeller nærmet seg en realitet!
