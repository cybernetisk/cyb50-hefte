\chapter{Kvinneandelen på Ifi}

\label{chap:kvinneandelen}

\author{Skrevet av Arne Hassel, med god hjelp fra administrasjonen på Ifi ved Eli Berge}

Da vi satte oss fore å lage denne boken ble jeg spurt om vi kunne lage en oversikt over hvordan kvinneandelen har utviklet seg på Ifi, og blåøyd og naiv som jeg var tenkte jeg <<Det hadde vært kjempeinteressert, selvfølgelig skal vi få til det!>> 

Men hvordan måler man kvinneandelen på Ifi? Ser man på hvor mange kvinner som har vært tatt opp gjennom tidene? Ser man på hvor mange som har fullført studiene? Kanskje man heller skal se på studiepoengproduksjonen? Når man så har valgt hva man ønsker å måle, starter oppgaven med å samle data man kan bruke. Dette er ingen enkel affære, siden man har ikke har tilgang til rådataene (av åpenbare, personvernsensitive grunner), og så må man altså basere seg på aggregerte datasett som har blitt produsert opp gjennom tidene. Disse datasettene ligger i forskjellige databaser og gjerne med forskjellig struktur opp gjennom årene, siden hva man velger å fokusere på har endret seg.

Som du skjønner, kjære leser, så var dette ingen enkel jobb. Heldigvis fikk jeg god hjelp fra adminstrasjonen på Ifi, som også jobber med å få denne statistikken tilgjengelig på en forståelige måte\footnote{Noe av arbeidet deres kan finnes på \url{https://www.mn.uio.no/ifi/studier/jenter-og-informatikk/statistikk/}}.

Tabell \ref{table:kvinneandel} viser det samlede opptaket registrert på Ifi siden 1998, og videre følger figur \ref{figure:kvinneandel} som viser utviklingen visuelt. Grunnen til at utviklingen svinger såpass som den gjør de første årene har mye med å gjøre at omstrukturering av studieprogrammet på Ifi og samfunnsmessige endringer gir utslag som ikke kan forklares via en graf.

På grunn av alle faktorene som spiller inn i slik statistikk, så er det alltid vanskelig å konkludere med noe konkret. Men det \textit{kan} se ut som at kvinneandelen går stødt og stadig oppover, spesielt etter at studieprogrammene stabiliserte seg mer i 2010 og utover. Om dette er noe som vil fortsette er vanskelig å si, men det er lov å håpe at flere får øynene opp for hvor viktig informatikken er i samfunnet. Det er også lov å håpe at de som allerede er en del av miljøet klarer å inkludere de nye generasjonene med informatikere som utdanner seg.

\begin{table}[]
	\begin{tabular}{|r|r|r|r|r|r|r|r|}
		\cline{2-7}
		\multicolumn{1}{r|}{} & \multicolumn{2}{c|}{Laveregrad} & \multicolumn{2}{c|}{Master}  & \multicolumn{2}{c|}{Profesjons} & \multicolumn{1}{r}{} \\
		\cline{2-8}
		\multicolumn{1}{r|}{} & Kvinner & Menn & Kvinner & Menn & Kvinner & Menn & Kvinner \% \\
		\hline
		1998 & & & 10 & 105 & & & 8,7\% \\
		\hline
		1999 & & & 25 & 85 & & & 22,7\% \\
		\hline
		2000 & & & 15 & 125 & & & 10,7\% \\
		\hline
		2001 & & & 40 & 100 & 10 & 40 & 26,3\% \\
		\hline
		2002 & & & 25 & 135 & 15 & 70 & 16,3\% \\
		\hline
		2003 & 70 & 215 & 60 & 165 & 10 & 75 & 23,5\% \\
		\hline
		2004 & 70 & 270 & 45 & 150 & 5 & 40 & 20,7\% \\
		\hline
		2005 & 45 & 135 & 30 & 120 & 10 & 45 & 22,1\% \\
		\hline
		2006 & 60 & 185 & 30 & 130 & 10 & 50 & 21,5\% \\
		\hline
		2007 & 85 & 185 & 20 & 85 & 15 & 60 & 26,7\% \\
		\hline
		2008 & 75 & 190 & 20 & 70 & 5 & 55 & 24,1\% \\
		\hline
		2009 & 75 & 210 & 30 & 130 & 10 & 60 & 22,3\% \\
		\hline
		2010 & 65 & 305 & 45 & 145 & & & 19,6\% \\
		\hline
		2011 & 85 & 310 & 35 & 140 & & & 21,1\% \\
		\hline
		2012 & 75 & 310 & 50 & 155 & & & 21,2\% \\
		\hline
		2013 & 100 & 345 & 40 & 200 & & & 20,4\% \\
		\hline
		2014 & 85 & 310 & 40 & 200 & & & 19,7\% \\
		\hline
		2015 & 95 & 340 & 55 & 205 & & & 21,6\% \\
		\hline
		2016 & 130 & 350 & 60 & 220 & & & 25,0\% \\
		\hline
		2017 & 155 & 340 & 75 & 270 & & & 27,4\% \\
		\hline
		2018 & 165 & 315 & 55 & 195 & & & 30,1\% \\
		\hline
	\end{tabular}
	\caption{Opptak av studenter til Ifi 1998--2018}
	\label{table:kvinneandel}
\end{table}

\begin{figure}
	Her kommer det graf...
	% TODO: Vente på hjelp fra Veronika til graf
	\caption{Graf som viser kvinneandelen over tid i opptaket til Ifi}
	\label{figure:kvinneandel}
\end{figure}

Det skal nevnes at jeg forsøkte å finne frem tall fra Samordna opptak i håp om å finne ut hvordan fordelingen av kjønn har utviklet seg for søkingen til studiene, men her støtte jeg på problemet med datasett som har ulik struktur, og viser forskjellige ting. Ikke minst har studieprogrammene som sagt utviklet seg gjennom tiden, og jeg fant ikke søkertall opp til lavere- og høyeregrads studier. Så da ble det med tallene jeg fikk fra administrasjonen.

Om andre ønsker å lage seg en oversikt over utviklingen av kvinneandelen på Ifi ønsker jeg dem lykke til. Noen kilder som kan være av interesse er databasen for statistikk om høyere utdanning (DBH)\footnote{\url{https://dbh.nsd.uib.no/statistikk/}} og Samordna opptak (SO) sin statistikk over søker- og opptakstall\footnote{\url{https://www.samordnaopptak.no/info/om/sokertall/}}. Det kan også hende at administrasjonen på Ifi kan hjelpe deg med å hente ut tall fra deres Felles studentsystem (FS)\footnote{\url{https://www.uio.no/for-ansatte/arbeidsstotte/sta/fs/statistikk/}}. Merk at tallene fra FS, som er brukt i tabellen og figuren her, vil være annerledes enn tallene fra DBH siden tallene for FS tar høyde for registreringer etter rapporteringsfristen.