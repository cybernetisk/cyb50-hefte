\chapter*{Introduksjon}

\setcounter{footnote}{0}

I denne bokens første del tar vi for oss den overordnete historien til CYB og Ifi. Vi har valgt å starte delen med to kapitler som er gjenbrukt fra jubileumsheftet fra CYB 25-årsfeiring, da disse meget godt beskriver CYB sin spede begynnelse.

Videre har vi fått med oss Narve Trædal til å skrive om instituttets historie, en historie han leder oss igjennom fra røttene på Matematisk og Fysisk institutt til innflyttingen på Ole-Johan Dahls hus. Innimellom her kan vi også lese om etableringen av Ifi og flytteprosessen til det som da var det nye instituttbygget i 1988.

Men hovedfokuset ligger på CYB, og kanskje spesielt den nyere tiden. Tiden i forkant av Ole-Johan Dahls hus, arbeidet med å få på plass en ny studentkjeller, engasjementet og profesjonaliseringen av driften, samy CYBs plass i det yrende studentmiljøet, både på og utenfor Gaustadalléen 23B.

Det er mye å lære av historien, og vi håper teksene i denne boken kan gi noen innsikter man kan ta med seg videre. Det er også mye å være stolt av i CYBs historie, og vi håper tekstene kan være med i fundamentet for stoltheten man med berretigelse kan ha for studentforeningen som i år er 50 år.